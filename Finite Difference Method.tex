\documentclass[11pt]{article}

    \usepackage[breakable]{tcolorbox}
    \usepackage{parskip} % Stop auto-indenting (to mimic markdown behaviour)
    

    % Basic figure setup, for now with no caption control since it's done
    % automatically by Pandoc (which extracts ![](path) syntax from Markdown).
    \usepackage{graphicx}
    % Maintain compatibility with old templates. Remove in nbconvert 6.0
    \let\Oldincludegraphics\includegraphics
    % Ensure that by default, figures have no caption (until we provide a
    % proper Figure object with a Caption API and a way to capture that
    % in the conversion process - todo).
    \usepackage{caption}
    \DeclareCaptionFormat{nocaption}{}
    \captionsetup{format=nocaption,aboveskip=0pt,belowskip=0pt}

    \usepackage{float}
    \floatplacement{figure}{H} % forces figures to be placed at the correct location
    \usepackage{xcolor} % Allow colors to be defined
    \usepackage{enumerate} % Needed for markdown enumerations to work
    \usepackage{geometry} % Used to adjust the document margins
    \usepackage{amsmath} % Equations
    \usepackage{amssymb} % Equations
    \usepackage{textcomp} % defines textquotesingle
    % Hack from http://tex.stackexchange.com/a/47451/13684:
    \AtBeginDocument{%
        \def\PYZsq{\textquotesingle}% Upright quotes in Pygmentized code
    }
    \usepackage{upquote} % Upright quotes for verbatim code
    \usepackage{eurosym} % defines \euro

    \usepackage{iftex}
    \ifPDFTeX
        \usepackage[T1]{fontenc}
        \IfFileExists{alphabeta.sty}{
              \usepackage{alphabeta}
          }{
              \usepackage[mathletters]{ucs}
              \usepackage[utf8x]{inputenc}
          }
    \else
        \usepackage{fontspec}
        \usepackage{unicode-math}
    \fi

    \usepackage{fancyvrb} % verbatim replacement that allows latex
    \usepackage{grffile} % extends the file name processing of package graphics 
                         % to support a larger range
    \makeatletter % fix for old versions of grffile with XeLaTeX
    \@ifpackagelater{grffile}{2019/11/01}
    {
      % Do nothing on new versions
    }
    {
      \def\Gread@@xetex#1{%
        \IfFileExists{"\Gin@base".bb}%
        {\Gread@eps{\Gin@base.bb}}%
        {\Gread@@xetex@aux#1}%
      }
    }
    \makeatother
    \usepackage[Export]{adjustbox} % Used to constrain images to a maximum size
    \adjustboxset{max size={0.9\linewidth}{0.9\paperheight}}

    % The hyperref package gives us a pdf with properly built
    % internal navigation ('pdf bookmarks' for the table of contents,
    % internal cross-reference links, web links for URLs, etc.)
    \usepackage{hyperref}
    % The default LaTeX title has an obnoxious amount of whitespace. By default,
    % titling removes some of it. It also provides customization options.
    \usepackage{titling}
    \usepackage{longtable} % longtable support required by pandoc >1.10
    \usepackage{booktabs}  % table support for pandoc > 1.12.2
    \usepackage{array}     % table support for pandoc >= 2.11.3
    \usepackage{calc}      % table minipage width calculation for pandoc >= 2.11.1
    \usepackage[inline]{enumitem} % IRkernel/repr support (it uses the enumerate* environment)
    \usepackage[normalem]{ulem} % ulem is needed to support strikethroughs (\sout)
                                % normalem makes italics be italics, not underlines
    \usepackage{mathrsfs}
    

    
    % Colors for the hyperref package
    \definecolor{urlcolor}{rgb}{0,.145,.698}
    \definecolor{linkcolor}{rgb}{.71,0.21,0.01}
    \definecolor{citecolor}{rgb}{.12,.54,.11}

    % ANSI colors
    \definecolor{ansi-black}{HTML}{3E424D}
    \definecolor{ansi-black-intense}{HTML}{282C36}
    \definecolor{ansi-red}{HTML}{E75C58}
    \definecolor{ansi-red-intense}{HTML}{B22B31}
    \definecolor{ansi-green}{HTML}{00A250}
    \definecolor{ansi-green-intense}{HTML}{007427}
    \definecolor{ansi-yellow}{HTML}{DDB62B}
    \definecolor{ansi-yellow-intense}{HTML}{B27D12}
    \definecolor{ansi-blue}{HTML}{208FFB}
    \definecolor{ansi-blue-intense}{HTML}{0065CA}
    \definecolor{ansi-magenta}{HTML}{D160C4}
    \definecolor{ansi-magenta-intense}{HTML}{A03196}
    \definecolor{ansi-cyan}{HTML}{60C6C8}
    \definecolor{ansi-cyan-intense}{HTML}{258F8F}
    \definecolor{ansi-white}{HTML}{C5C1B4}
    \definecolor{ansi-white-intense}{HTML}{A1A6B2}
    \definecolor{ansi-default-inverse-fg}{HTML}{FFFFFF}
    \definecolor{ansi-default-inverse-bg}{HTML}{000000}

    % common color for the border for error outputs.
    \definecolor{outerrorbackground}{HTML}{FFDFDF}

    % commands and environments needed by pandoc snippets
    % extracted from the output of `pandoc -s`
    \providecommand{\tightlist}{%
      \setlength{\itemsep}{0pt}\setlength{\parskip}{0pt}}
    \DefineVerbatimEnvironment{Highlighting}{Verbatim}{commandchars=\\\{\}}
    % Add ',fontsize=\small' for more characters per line
    \newenvironment{Shaded}{}{}
    \newcommand{\KeywordTok}[1]{\textcolor[rgb]{0.00,0.44,0.13}{\textbf{{#1}}}}
    \newcommand{\DataTypeTok}[1]{\textcolor[rgb]{0.56,0.13,0.00}{{#1}}}
    \newcommand{\DecValTok}[1]{\textcolor[rgb]{0.25,0.63,0.44}{{#1}}}
    \newcommand{\BaseNTok}[1]{\textcolor[rgb]{0.25,0.63,0.44}{{#1}}}
    \newcommand{\FloatTok}[1]{\textcolor[rgb]{0.25,0.63,0.44}{{#1}}}
    \newcommand{\CharTok}[1]{\textcolor[rgb]{0.25,0.44,0.63}{{#1}}}
    \newcommand{\StringTok}[1]{\textcolor[rgb]{0.25,0.44,0.63}{{#1}}}
    \newcommand{\CommentTok}[1]{\textcolor[rgb]{0.38,0.63,0.69}{\textit{{#1}}}}
    \newcommand{\OtherTok}[1]{\textcolor[rgb]{0.00,0.44,0.13}{{#1}}}
    \newcommand{\AlertTok}[1]{\textcolor[rgb]{1.00,0.00,0.00}{\textbf{{#1}}}}
    \newcommand{\FunctionTok}[1]{\textcolor[rgb]{0.02,0.16,0.49}{{#1}}}
    \newcommand{\RegionMarkerTok}[1]{{#1}}
    \newcommand{\ErrorTok}[1]{\textcolor[rgb]{1.00,0.00,0.00}{\textbf{{#1}}}}
    \newcommand{\NormalTok}[1]{{#1}}
    
    % Additional commands for more recent versions of Pandoc
    \newcommand{\ConstantTok}[1]{\textcolor[rgb]{0.53,0.00,0.00}{{#1}}}
    \newcommand{\SpecialCharTok}[1]{\textcolor[rgb]{0.25,0.44,0.63}{{#1}}}
    \newcommand{\VerbatimStringTok}[1]{\textcolor[rgb]{0.25,0.44,0.63}{{#1}}}
    \newcommand{\SpecialStringTok}[1]{\textcolor[rgb]{0.73,0.40,0.53}{{#1}}}
    \newcommand{\ImportTok}[1]{{#1}}
    \newcommand{\DocumentationTok}[1]{\textcolor[rgb]{0.73,0.13,0.13}{\textit{{#1}}}}
    \newcommand{\AnnotationTok}[1]{\textcolor[rgb]{0.38,0.63,0.69}{\textbf{\textit{{#1}}}}}
    \newcommand{\CommentVarTok}[1]{\textcolor[rgb]{0.38,0.63,0.69}{\textbf{\textit{{#1}}}}}
    \newcommand{\VariableTok}[1]{\textcolor[rgb]{0.10,0.09,0.49}{{#1}}}
    \newcommand{\ControlFlowTok}[1]{\textcolor[rgb]{0.00,0.44,0.13}{\textbf{{#1}}}}
    \newcommand{\OperatorTok}[1]{\textcolor[rgb]{0.40,0.40,0.40}{{#1}}}
    \newcommand{\BuiltInTok}[1]{{#1}}
    \newcommand{\ExtensionTok}[1]{{#1}}
    \newcommand{\PreprocessorTok}[1]{\textcolor[rgb]{0.74,0.48,0.00}{{#1}}}
    \newcommand{\AttributeTok}[1]{\textcolor[rgb]{0.49,0.56,0.16}{{#1}}}
    \newcommand{\InformationTok}[1]{\textcolor[rgb]{0.38,0.63,0.69}{\textbf{\textit{{#1}}}}}
    \newcommand{\WarningTok}[1]{\textcolor[rgb]{0.38,0.63,0.69}{\textbf{\textit{{#1}}}}}
    
    
    % Define a nice break command that doesn't care if a line doesn't already
    % exist.
    \def\br{\hspace*{\fill} \\* }
    % Math Jax compatibility definitions
    \def\gt{>}
    \def\lt{<}
    \let\Oldtex\TeX
    \let\Oldlatex\LaTeX
    \renewcommand{\TeX}{\textrm{\Oldtex}}
    \renewcommand{\LaTeX}{\textrm{\Oldlatex}}
    % Document parameters
    % Document title
    \title{Finite Difference Method}
    
    
    
    
    
% Pygments definitions
\makeatletter
\def\PY@reset{\let\PY@it=\relax \let\PY@bf=\relax%
    \let\PY@ul=\relax \let\PY@tc=\relax%
    \let\PY@bc=\relax \let\PY@ff=\relax}
\def\PY@tok#1{\csname PY@tok@#1\endcsname}
\def\PY@toks#1+{\ifx\relax#1\empty\else%
    \PY@tok{#1}\expandafter\PY@toks\fi}
\def\PY@do#1{\PY@bc{\PY@tc{\PY@ul{%
    \PY@it{\PY@bf{\PY@ff{#1}}}}}}}
\def\PY#1#2{\PY@reset\PY@toks#1+\relax+\PY@do{#2}}

\@namedef{PY@tok@w}{\def\PY@tc##1{\textcolor[rgb]{0.73,0.73,0.73}{##1}}}
\@namedef{PY@tok@c}{\let\PY@it=\textit\def\PY@tc##1{\textcolor[rgb]{0.25,0.50,0.50}{##1}}}
\@namedef{PY@tok@cp}{\def\PY@tc##1{\textcolor[rgb]{0.74,0.48,0.00}{##1}}}
\@namedef{PY@tok@k}{\let\PY@bf=\textbf\def\PY@tc##1{\textcolor[rgb]{0.00,0.50,0.00}{##1}}}
\@namedef{PY@tok@kp}{\def\PY@tc##1{\textcolor[rgb]{0.00,0.50,0.00}{##1}}}
\@namedef{PY@tok@kt}{\def\PY@tc##1{\textcolor[rgb]{0.69,0.00,0.25}{##1}}}
\@namedef{PY@tok@o}{\def\PY@tc##1{\textcolor[rgb]{0.40,0.40,0.40}{##1}}}
\@namedef{PY@tok@ow}{\let\PY@bf=\textbf\def\PY@tc##1{\textcolor[rgb]{0.67,0.13,1.00}{##1}}}
\@namedef{PY@tok@nb}{\def\PY@tc##1{\textcolor[rgb]{0.00,0.50,0.00}{##1}}}
\@namedef{PY@tok@nf}{\def\PY@tc##1{\textcolor[rgb]{0.00,0.00,1.00}{##1}}}
\@namedef{PY@tok@nc}{\let\PY@bf=\textbf\def\PY@tc##1{\textcolor[rgb]{0.00,0.00,1.00}{##1}}}
\@namedef{PY@tok@nn}{\let\PY@bf=\textbf\def\PY@tc##1{\textcolor[rgb]{0.00,0.00,1.00}{##1}}}
\@namedef{PY@tok@ne}{\let\PY@bf=\textbf\def\PY@tc##1{\textcolor[rgb]{0.82,0.25,0.23}{##1}}}
\@namedef{PY@tok@nv}{\def\PY@tc##1{\textcolor[rgb]{0.10,0.09,0.49}{##1}}}
\@namedef{PY@tok@no}{\def\PY@tc##1{\textcolor[rgb]{0.53,0.00,0.00}{##1}}}
\@namedef{PY@tok@nl}{\def\PY@tc##1{\textcolor[rgb]{0.63,0.63,0.00}{##1}}}
\@namedef{PY@tok@ni}{\let\PY@bf=\textbf\def\PY@tc##1{\textcolor[rgb]{0.60,0.60,0.60}{##1}}}
\@namedef{PY@tok@na}{\def\PY@tc##1{\textcolor[rgb]{0.49,0.56,0.16}{##1}}}
\@namedef{PY@tok@nt}{\let\PY@bf=\textbf\def\PY@tc##1{\textcolor[rgb]{0.00,0.50,0.00}{##1}}}
\@namedef{PY@tok@nd}{\def\PY@tc##1{\textcolor[rgb]{0.67,0.13,1.00}{##1}}}
\@namedef{PY@tok@s}{\def\PY@tc##1{\textcolor[rgb]{0.73,0.13,0.13}{##1}}}
\@namedef{PY@tok@sd}{\let\PY@it=\textit\def\PY@tc##1{\textcolor[rgb]{0.73,0.13,0.13}{##1}}}
\@namedef{PY@tok@si}{\let\PY@bf=\textbf\def\PY@tc##1{\textcolor[rgb]{0.73,0.40,0.53}{##1}}}
\@namedef{PY@tok@se}{\let\PY@bf=\textbf\def\PY@tc##1{\textcolor[rgb]{0.73,0.40,0.13}{##1}}}
\@namedef{PY@tok@sr}{\def\PY@tc##1{\textcolor[rgb]{0.73,0.40,0.53}{##1}}}
\@namedef{PY@tok@ss}{\def\PY@tc##1{\textcolor[rgb]{0.10,0.09,0.49}{##1}}}
\@namedef{PY@tok@sx}{\def\PY@tc##1{\textcolor[rgb]{0.00,0.50,0.00}{##1}}}
\@namedef{PY@tok@m}{\def\PY@tc##1{\textcolor[rgb]{0.40,0.40,0.40}{##1}}}
\@namedef{PY@tok@gh}{\let\PY@bf=\textbf\def\PY@tc##1{\textcolor[rgb]{0.00,0.00,0.50}{##1}}}
\@namedef{PY@tok@gu}{\let\PY@bf=\textbf\def\PY@tc##1{\textcolor[rgb]{0.50,0.00,0.50}{##1}}}
\@namedef{PY@tok@gd}{\def\PY@tc##1{\textcolor[rgb]{0.63,0.00,0.00}{##1}}}
\@namedef{PY@tok@gi}{\def\PY@tc##1{\textcolor[rgb]{0.00,0.63,0.00}{##1}}}
\@namedef{PY@tok@gr}{\def\PY@tc##1{\textcolor[rgb]{1.00,0.00,0.00}{##1}}}
\@namedef{PY@tok@ge}{\let\PY@it=\textit}
\@namedef{PY@tok@gs}{\let\PY@bf=\textbf}
\@namedef{PY@tok@gp}{\let\PY@bf=\textbf\def\PY@tc##1{\textcolor[rgb]{0.00,0.00,0.50}{##1}}}
\@namedef{PY@tok@go}{\def\PY@tc##1{\textcolor[rgb]{0.53,0.53,0.53}{##1}}}
\@namedef{PY@tok@gt}{\def\PY@tc##1{\textcolor[rgb]{0.00,0.27,0.87}{##1}}}
\@namedef{PY@tok@err}{\def\PY@bc##1{{\setlength{\fboxsep}{\string -\fboxrule}\fcolorbox[rgb]{1.00,0.00,0.00}{1,1,1}{\strut ##1}}}}
\@namedef{PY@tok@kc}{\let\PY@bf=\textbf\def\PY@tc##1{\textcolor[rgb]{0.00,0.50,0.00}{##1}}}
\@namedef{PY@tok@kd}{\let\PY@bf=\textbf\def\PY@tc##1{\textcolor[rgb]{0.00,0.50,0.00}{##1}}}
\@namedef{PY@tok@kn}{\let\PY@bf=\textbf\def\PY@tc##1{\textcolor[rgb]{0.00,0.50,0.00}{##1}}}
\@namedef{PY@tok@kr}{\let\PY@bf=\textbf\def\PY@tc##1{\textcolor[rgb]{0.00,0.50,0.00}{##1}}}
\@namedef{PY@tok@bp}{\def\PY@tc##1{\textcolor[rgb]{0.00,0.50,0.00}{##1}}}
\@namedef{PY@tok@fm}{\def\PY@tc##1{\textcolor[rgb]{0.00,0.00,1.00}{##1}}}
\@namedef{PY@tok@vc}{\def\PY@tc##1{\textcolor[rgb]{0.10,0.09,0.49}{##1}}}
\@namedef{PY@tok@vg}{\def\PY@tc##1{\textcolor[rgb]{0.10,0.09,0.49}{##1}}}
\@namedef{PY@tok@vi}{\def\PY@tc##1{\textcolor[rgb]{0.10,0.09,0.49}{##1}}}
\@namedef{PY@tok@vm}{\def\PY@tc##1{\textcolor[rgb]{0.10,0.09,0.49}{##1}}}
\@namedef{PY@tok@sa}{\def\PY@tc##1{\textcolor[rgb]{0.73,0.13,0.13}{##1}}}
\@namedef{PY@tok@sb}{\def\PY@tc##1{\textcolor[rgb]{0.73,0.13,0.13}{##1}}}
\@namedef{PY@tok@sc}{\def\PY@tc##1{\textcolor[rgb]{0.73,0.13,0.13}{##1}}}
\@namedef{PY@tok@dl}{\def\PY@tc##1{\textcolor[rgb]{0.73,0.13,0.13}{##1}}}
\@namedef{PY@tok@s2}{\def\PY@tc##1{\textcolor[rgb]{0.73,0.13,0.13}{##1}}}
\@namedef{PY@tok@sh}{\def\PY@tc##1{\textcolor[rgb]{0.73,0.13,0.13}{##1}}}
\@namedef{PY@tok@s1}{\def\PY@tc##1{\textcolor[rgb]{0.73,0.13,0.13}{##1}}}
\@namedef{PY@tok@mb}{\def\PY@tc##1{\textcolor[rgb]{0.40,0.40,0.40}{##1}}}
\@namedef{PY@tok@mf}{\def\PY@tc##1{\textcolor[rgb]{0.40,0.40,0.40}{##1}}}
\@namedef{PY@tok@mh}{\def\PY@tc##1{\textcolor[rgb]{0.40,0.40,0.40}{##1}}}
\@namedef{PY@tok@mi}{\def\PY@tc##1{\textcolor[rgb]{0.40,0.40,0.40}{##1}}}
\@namedef{PY@tok@il}{\def\PY@tc##1{\textcolor[rgb]{0.40,0.40,0.40}{##1}}}
\@namedef{PY@tok@mo}{\def\PY@tc##1{\textcolor[rgb]{0.40,0.40,0.40}{##1}}}
\@namedef{PY@tok@ch}{\let\PY@it=\textit\def\PY@tc##1{\textcolor[rgb]{0.25,0.50,0.50}{##1}}}
\@namedef{PY@tok@cm}{\let\PY@it=\textit\def\PY@tc##1{\textcolor[rgb]{0.25,0.50,0.50}{##1}}}
\@namedef{PY@tok@cpf}{\let\PY@it=\textit\def\PY@tc##1{\textcolor[rgb]{0.25,0.50,0.50}{##1}}}
\@namedef{PY@tok@c1}{\let\PY@it=\textit\def\PY@tc##1{\textcolor[rgb]{0.25,0.50,0.50}{##1}}}
\@namedef{PY@tok@cs}{\let\PY@it=\textit\def\PY@tc##1{\textcolor[rgb]{0.25,0.50,0.50}{##1}}}

\def\PYZbs{\char`\\}
\def\PYZus{\char`\_}
\def\PYZob{\char`\{}
\def\PYZcb{\char`\}}
\def\PYZca{\char`\^}
\def\PYZam{\char`\&}
\def\PYZlt{\char`\<}
\def\PYZgt{\char`\>}
\def\PYZsh{\char`\#}
\def\PYZpc{\char`\%}
\def\PYZdl{\char`\$}
\def\PYZhy{\char`\-}
\def\PYZsq{\char`\'}
\def\PYZdq{\char`\"}
\def\PYZti{\char`\~}
% for compatibility with earlier versions
\def\PYZat{@}
\def\PYZlb{[}
\def\PYZrb{]}
\makeatother


    % For linebreaks inside Verbatim environment from package fancyvrb. 
    \makeatletter
        \newbox\Wrappedcontinuationbox 
        \newbox\Wrappedvisiblespacebox 
        \newcommand*\Wrappedvisiblespace {\textcolor{red}{\textvisiblespace}} 
        \newcommand*\Wrappedcontinuationsymbol {\textcolor{red}{\llap{\tiny$\m@th\hookrightarrow$}}} 
        \newcommand*\Wrappedcontinuationindent {3ex } 
        \newcommand*\Wrappedafterbreak {\kern\Wrappedcontinuationindent\copy\Wrappedcontinuationbox} 
        % Take advantage of the already applied Pygments mark-up to insert 
        % potential linebreaks for TeX processing. 
        %        {, <, #, %, $, ' and ": go to next line. 
        %        _, }, ^, &, >, - and ~: stay at end of broken line. 
        % Use of \textquotesingle for straight quote. 
        \newcommand*\Wrappedbreaksatspecials {% 
            \def\PYGZus{\discretionary{\char`\_}{\Wrappedafterbreak}{\char`\_}}% 
            \def\PYGZob{\discretionary{}{\Wrappedafterbreak\char`\{}{\char`\{}}% 
            \def\PYGZcb{\discretionary{\char`\}}{\Wrappedafterbreak}{\char`\}}}% 
            \def\PYGZca{\discretionary{\char`\^}{\Wrappedafterbreak}{\char`\^}}% 
            \def\PYGZam{\discretionary{\char`\&}{\Wrappedafterbreak}{\char`\&}}% 
            \def\PYGZlt{\discretionary{}{\Wrappedafterbreak\char`\<}{\char`\<}}% 
            \def\PYGZgt{\discretionary{\char`\>}{\Wrappedafterbreak}{\char`\>}}% 
            \def\PYGZsh{\discretionary{}{\Wrappedafterbreak\char`\#}{\char`\#}}% 
            \def\PYGZpc{\discretionary{}{\Wrappedafterbreak\char`\%}{\char`\%}}% 
            \def\PYGZdl{\discretionary{}{\Wrappedafterbreak\char`\$}{\char`\$}}% 
            \def\PYGZhy{\discretionary{\char`\-}{\Wrappedafterbreak}{\char`\-}}% 
            \def\PYGZsq{\discretionary{}{\Wrappedafterbreak\textquotesingle}{\textquotesingle}}% 
            \def\PYGZdq{\discretionary{}{\Wrappedafterbreak\char`\"}{\char`\"}}% 
            \def\PYGZti{\discretionary{\char`\~}{\Wrappedafterbreak}{\char`\~}}% 
        } 
        % Some characters . , ; ? ! / are not pygmentized. 
        % This macro makes them "active" and they will insert potential linebreaks 
        \newcommand*\Wrappedbreaksatpunct {% 
            \lccode`\~`\.\lowercase{\def~}{\discretionary{\hbox{\char`\.}}{\Wrappedafterbreak}{\hbox{\char`\.}}}% 
            \lccode`\~`\,\lowercase{\def~}{\discretionary{\hbox{\char`\,}}{\Wrappedafterbreak}{\hbox{\char`\,}}}% 
            \lccode`\~`\;\lowercase{\def~}{\discretionary{\hbox{\char`\;}}{\Wrappedafterbreak}{\hbox{\char`\;}}}% 
            \lccode`\~`\:\lowercase{\def~}{\discretionary{\hbox{\char`\:}}{\Wrappedafterbreak}{\hbox{\char`\:}}}% 
            \lccode`\~`\?\lowercase{\def~}{\discretionary{\hbox{\char`\?}}{\Wrappedafterbreak}{\hbox{\char`\?}}}% 
            \lccode`\~`\!\lowercase{\def~}{\discretionary{\hbox{\char`\!}}{\Wrappedafterbreak}{\hbox{\char`\!}}}% 
            \lccode`\~`\/\lowercase{\def~}{\discretionary{\hbox{\char`\/}}{\Wrappedafterbreak}{\hbox{\char`\/}}}% 
            \catcode`\.\active
            \catcode`\,\active 
            \catcode`\;\active
            \catcode`\:\active
            \catcode`\?\active
            \catcode`\!\active
            \catcode`\/\active 
            \lccode`\~`\~ 	
        }
    \makeatother

    \let\OriginalVerbatim=\Verbatim
    \makeatletter
    \renewcommand{\Verbatim}[1][1]{%
        %\parskip\z@skip
        \sbox\Wrappedcontinuationbox {\Wrappedcontinuationsymbol}%
        \sbox\Wrappedvisiblespacebox {\FV@SetupFont\Wrappedvisiblespace}%
        \def\FancyVerbFormatLine ##1{\hsize\linewidth
            \vtop{\raggedright\hyphenpenalty\z@\exhyphenpenalty\z@
                \doublehyphendemerits\z@\finalhyphendemerits\z@
                \strut ##1\strut}%
        }%
        % If the linebreak is at a space, the latter will be displayed as visible
        % space at end of first line, and a continuation symbol starts next line.
        % Stretch/shrink are however usually zero for typewriter font.
        \def\FV@Space {%
            \nobreak\hskip\z@ plus\fontdimen3\font minus\fontdimen4\font
            \discretionary{\copy\Wrappedvisiblespacebox}{\Wrappedafterbreak}
            {\kern\fontdimen2\font}%
        }%
        
        % Allow breaks at special characters using \PYG... macros.
        \Wrappedbreaksatspecials
        % Breaks at punctuation characters . , ; ? ! and / need catcode=\active 	
        \OriginalVerbatim[#1,codes*=\Wrappedbreaksatpunct]%
    }
    \makeatother

    % Exact colors from NB
    \definecolor{incolor}{HTML}{303F9F}
    \definecolor{outcolor}{HTML}{D84315}
    \definecolor{cellborder}{HTML}{CFCFCF}
    \definecolor{cellbackground}{HTML}{F7F7F7}
    
    % prompt
    \makeatletter
    \newcommand{\boxspacing}{\kern\kvtcb@left@rule\kern\kvtcb@boxsep}
    \makeatother
    \newcommand{\prompt}[4]{
        {\ttfamily\llap{{\color{#2}[#3]:\hspace{3pt}#4}}\vspace{-\baselineskip}}
    }
    

    
    % Prevent overflowing lines due to hard-to-break entities
    \sloppy 
    % Setup hyperref package
    \hypersetup{
      breaklinks=true,  % so long urls are correctly broken across lines
      colorlinks=true,
      urlcolor=urlcolor,
      linkcolor=linkcolor,
      citecolor=citecolor,
      }
    % Slightly bigger margins than the latex defaults
    
    \geometry{verbose,tmargin=1in,bmargin=1in,lmargin=1in,rmargin=1in}
    
    

\begin{document}
    
    \maketitle
    
    

    
    \hypertarget{numerical-solutions}{%
\section{Numerical Solutions}\label{numerical-solutions}}

    Fun with numbers.

    \hypertarget{calculating-urho-phi-t-using-the-finite-difference-method}{%
\section{\texorpdfstring{Calculating u(\(\rho\), \(\phi\), t) Using the
Finite Difference
Method}{Calculating u(\textbackslash rho, \textbackslash phi, t) Using the Finite Difference Method}}\label{calculating-urho-phi-t-using-the-finite-difference-method}}

    The time-dependent diffusion equation can be solved in closed form.
However, it can be a challenge to use an infinite sum etc.
Alternaitvely, there are other methods that can be employed to calculate
the temperature at each point on the geometry over time. One of the
methods is called the finite difference method. This method is based on
using the Taylor series expansion of a fuction to calculate each
derivative order.

Starting with a fucntion f(x), the Taylor series expansion of this
function around a point \(x_0\) at a distance of \(\pm \Delta x\) from
\(x_0\) \begin{equation}
\begin{split}
f(x_0 + \Delta x) &= f(x_0) + (\Delta x) \frac{1}{1!} \frac{df(x_0)}{dx} + \frac{1}{2!}(\Delta x)^2 \frac{d^2f(x_0)}{dx^2} + \frac{1}{3!} (\Delta x)^3 \frac{d^3f(x_0)}{dx^3} + \frac{1}{4!} (\Delta x)^4 \frac{d^4f(x_0)}{dx^4} + \cdots \\
f(x_0 - \Delta x) &= f(x_0) + (-\Delta x) \frac{1}{1!} \frac{df(x_0)}{dx} + \frac{1}{2!}(-\Delta x)^2 \frac{d^2f(x_0)}{dx^2} + \frac{1}{3!} (-\Delta x)^3 \frac{d^3f(x_0)}{dx^3} + \frac{1}{4!} (-\Delta x)^4 \frac{d^4f(x_0)}{dx^4} + \cdots \\
&= f(x_0) - (\Delta x) \frac{1}{1!} \frac{df(x_0)}{dx} + \frac{1}{2!}(\Delta x)^2 \frac{d^2f(x_0)}{dx^2} - \frac{1}{3!} (\Delta x)^3 \frac{d^3f(x_0)}{dx^3} + \frac{1}{4!} (\Delta x)^4 \frac{d^4f(x_0)}{dx^4} + \cdots 
\end{split}
\end{equation}

The diffusion equation requires that the first and second-order
derivatives are calculated.\\
\begin{equation}
\nabla^2 u (a \hat {e_1}, b \hat {e_2}, c \hat {e_3}, t)   = \frac {\partial u}{\partial t}
\end{equation} In rectilinear coordinates: \begin{equation}
\frac {\partial^2 u}{\partial x^2} + \frac {\partial^2 u}{\partial y^2} = \frac {\partial u}{\partial t}
\end{equation}

and in cyclindrical coordinates: \begin{equation}
\frac {\partial^2 u}{\partial \rho^2} + \frac{1}{\rho} \frac {\partial u}{\partial \rho} + 
\frac{1}{\rho^2} \frac {\partial^2 u}{\partial \phi^2} = \frac {\partial u}{\partial t}
\end{equation}

The approximation for the first-order derivative can be found by taking
the difference \(f(x_0 + \Delta x) - f(x_0 - \Delta x)\)
\begin{equation}
\begin{split} 
f(x_0 + \Delta x) - f(x_0 - \Delta x) &= \left[f(x_0) + (\Delta x) \frac{df(x_0)}{dx} + \frac{1}{2!}(\Delta x)^2 \frac{d^2f(x_0)}{dx^2} + \frac{1}{3!} (\Delta x)^3 \frac{d^3f(x_0)}{dx^3} + \frac{1}{4!} (\Delta x)^4 \frac{d^4f(x_0)}{dx^4} + \cdots \right] - \left[ f(x_0) - (\Delta x) \frac{df(x_0)}{dx} + \frac{1}{2!}(\Delta x)^2 \frac{d^2f(x_0)}{dx^2} - \frac{1}{3!} (\Delta x)^3 \frac{d^3f(x_0)}{dx^3} + \frac{1}{4!} (\Delta x)^4 \frac{d^4f(x_0)}{dx^4} + \cdots \right] \\
&= 2 (\Delta x) \frac{df(x_0)}{dx} + \frac{2 (\Delta x)^3}{3!} \frac{d^3f(x_0)}{dx^3} + \cdots \\
&= 2 (\Delta x) \frac{df(x_0)}{dx} + \frac{(\Delta x)^3}{3} \frac{d^3f(x_0)}{dx^3} + \cdots 
\text{odd terms}
\end{split}
\end{equation}

If we assume that \(\Delta x \ll 1\), then we can assume that
\((\Delta x)^3 \ll 1\) and subsequent terms can be ignored. Having said
that, we will need to keep in mind that since we are approximating the
value of the derivative, there will be error terms to consider to ensure
that the final numerical solution is guaranteed to converge. More on
this later. \begin{equation}
\begin{split} 
f(x_0 + \Delta x) - f(x_0 - \Delta x) &= 2 (\Delta x) \frac{df(x_0)}{dx} + O((\Delta x)^3) \\
\frac{f(x_0 + \Delta x) - f(x_0 - \Delta x)}{2} &= (\Delta x) \frac{df(x_0)}{dx} + O((\Delta x)^3)  \\
\frac{f(x_0 + \Delta x) - f(x_0 - \Delta x)}{2} &\approx (\Delta x) \frac{df(x_0)}{dx} \\
\frac{df(x_0)}{dx} &\approx \frac{f(x_0 + \Delta x) - f(x_0 - \Delta x)}{2 (\Delta x)}
\end{split}
\end{equation}

The approximation for the second-order derivative can be found by taking
the sum \(f(x_0 + \Delta x) + f(x_0 - \Delta x)\) \begin{equation}
\begin{split} 
f(x_0 + \Delta x) + f(x_0 - \Delta x) &= \left[f(x_0) + (\Delta x) \frac{df(x_0)}{dx} + \frac{1}{2!}(\Delta x)^2 \frac{d^2f(x_0)}{dx^2} + \frac{1}{3!} (\Delta x)^3 \frac{d^3f(x_0)}{dx^3} + \frac{1}{4!} (\Delta x)^4 \frac{d^4f(x_0)}{dx^4} + \cdots \right] + \left[ f(x_0) - (\Delta x) \frac{df(x_0)}{dx} + \frac{1}{2!}(\Delta x)^2 \frac{d^2f(x_0)}{dx^2} - \frac{1}{3!} (\Delta x)^3 \frac{d^3f(x_0)}{dx^3} + \frac{1}{4!} (\Delta x)^4 \frac{d^4f(x_0)}{dx^4} + \cdots \right] \\
&= 2 f(x_0) + 2 (\frac{\Delta x^2}{2!}) \frac{d^2f(x_0)}{dx^2} + 2 (\frac{\Delta x^4}{4!}) \frac{d^4f(x_0)}{dx^4} + \cdots \text{even terms}
\end{split}
\end{equation}

Imposing the assumption that \(\Delta x \ll 1\), then we can assume that
\((\Delta x)^2 \ll 1\) and subsequent terms can be ignored.\\
\begin{equation}
\begin{split} 
f(x_0 + \Delta x) + f(x_0 - \Delta x) &= 2 f(x_0) + 2 (\frac{(\Delta x)^2}{2}) \frac{d^2f(x_0)}{dx^2} + O({\Delta x^4}) \\
f(x_0 + \Delta x) + f(x_0 - \Delta x) &\approx 2 f(x_0) + 2 (\frac{(\Delta x)^2}{2}) \frac{d^2f(x_0)}{dx^2} \\
f(x_0 + \Delta x) + f(x_0 - \Delta x) - 2 f(x_0) &\approx (\Delta x)^2 \frac{d^2f(x_0)}{dx^2} \\
\Delta x^2 \frac{d^2f(x_0)}{dx^2} &\approx f(x_0 + \Delta x) + f(x_0 - \Delta x) - 2 f(x_0)  \\
\frac{d^2f(x_0)}{dx^2} &\approx \frac{f(x_0 + \Delta x) + f(x_0 - \Delta x) - 2 f(x_0)}{(\Delta x)^2}
\end{split}
\end{equation}

The next question is how do we use these results to calculate the
temperature U at each point on the surface for each point in time. Since
we have the relationship to approximate the first-order and the
second-order derivative, we will need a way to distinguish each
coordinate value and time. Since this is step is arbitrary, we will set
the index i to represent the first coordinate (x or \(\rho\)), j to
represent the second coordinate (y or \(\phi\)) and m represent some
time increment.

In rectilinear coordinates:

A surface with sides X and Y and a time interval of T. The side X is
divided into P segments, Y is divided into Q segments and the total time
interval T is divided into R intervals: \begin{equation}
\begin{split} 
\Delta x &= \frac{X}{P} \\
\Delta y &= \frac{Y}{Q} \\
\Delta t &= \frac{T}{R}
\end{split}
\end{equation}

and \begin{equation}
\begin{split} 
i \in \{0, 1, 2, \cdots, P - 1 \} \\
j \in \{0, 1, 2, \cdots, Q - 1 \} \\
m \in \{0, 1, 2, \cdots, R - 1 \}
\end{split}
\end{equation}

\begin{equation}
\begin{split} 
u(x, y, t) &= u_{i,j}^m \\
u(x + \Delta x, y, t) &= u_{i + 1 ,j}^m \\
u(x - \Delta x, y, t) &= u_{i - 1 ,j}^m \\
u(x, y + \Delta y, t) &= u_{i,j + 1}^m \\
u(x, y - \Delta y, t) &= u_{i,j - 1}^m \\
u(x, y, t + \Delta t) &= u_{i,j}^{m + 1} \\
u(x, y, t - \Delta t) &= u_{i,j}^{m - 1} 
\end{split}
\end{equation}

In cylindrical coordinates: A surface with radius \(\rho\), angular
coordinate \(\phi\) and a time interval of T. The radius \(\rho\) is
divided into P segments, \(\phi\) is divided into Q segments and the
total time interval T is divided into R intervals: \begin{equation}
\begin{split} 
\Delta \rho &= \frac{\rho}{P} \\
\Delta \phi &= \frac{\phi}{Q} \\
\Delta t &= \frac{T}{R}
\end{split}
\end{equation}

and \begin{equation}
\begin{split} 
i \in \{0, 1, 2, \cdots, P - 1 \} \\
j \in \{0, 1, 2, \cdots, Q - 1 \} \\
m \in \{0, 1, 2, \cdots, R - 1 \}
\end{split}
\end{equation}

\begin{equation}
\begin{split} 
u(\rho, \phi, t) &= u_{i,j}^m \\
u(\rho + \Delta \rho, \phi, t)  &= u_{i + 1 ,j}^m \\
u(\rho - \Delta \rho, \phi, t) &= u_{i - 1 ,j}^m \\
u(\rho, y + \Delta \phi, t) &= u_{i,j + 1}^m \\
u(\rho, y - \Delta \phi, t) &= u_{i,j - 1}^m \\
u(\rho, y, t + \Delta t) &= u_{i,j}^{m + 1} \\
u(\rho, y, t - \Delta t) &= u_{i,j}^{m - 1} 
\end{split}
\end{equation}

    \hypertarget{numerical-solution-in-rectilinear-coordinates}{%
\subsection{Numerical Solution in Rectilinear
Coordinates}\label{numerical-solution-in-rectilinear-coordinates}}

    Recall the PDE for a 2-dimensional rectilinear surface: \begin{equation}
\frac {\partial^2 u}{\partial x^2} + \frac {\partial^2 u}{\partial y^2} = \frac {\partial u}{\partial t}
\end{equation}

NOTE: Centered in space, forward in time. Combinine equations xx and yy,
the finite difference method generates the following terms to replace
the partial derivatives: \begin{equation}
\begin{split} 
\frac{df}{dt} &\approx \frac{f(x, y, t + \Delta t) - f(x, y, t)}{2 (\Delta t)} \\
&= \frac{u_{i,j}^{m + 1} - u_{i,j}^{m}}{2 (\Delta t)} \\
\frac{d^2f}{dx^2} &\approx \frac{f(x + \Delta x, y, t) + f(x - \Delta x, y, t) - 2 f(x, y, t)}{(\Delta x)^2} \\
&= \frac{u_{i + 1 ,j}^m + u_{i - 1 ,j}^m - 2 u_{i,j}^m}{(\Delta x)^2} \\
\frac{d^2f}{dy^2} &\approx \frac{f(x, y + \Delta y, t) + f(x, y - \Delta y, t) - 2 f(x, y, t)}{(\Delta y)^2} \\
&= \frac{u_{i,j + 1}^m + u_{i,j - 1}^m - 2 u_{i,j}^m}{(\Delta y)^2}  
\end{split} 
\end{equation}

And the PDE becomes \begin{equation}
\begin{split}
\underbrace{\frac{u_{i,j}^{m + 1} - u_{i,j}^{m}}{2 (\Delta t)}}_{\frac {\partial u}{\partial t}} =\underbrace{\frac{u_{i + 1 ,j}^m + u_{i - 1 ,j}^m - 2 u_{i,j}^m}{(\Delta x)^2}}_{\frac {\partial^2 u}{\partial x^2}} + \underbrace{\frac{u_{i,j + 1}^m + u_{i,j - 1}^m - 2 u_{i,j}^m}{(\Delta y)^2}}_{\frac {\partial^2 u}{\partial y^2}} 
\end{split}
\end{equation}

The \(\Delta x\) and \(\Delta y\) terms are going to be small, but not
too small to ensure that amount of time and memory needed to perform the
calculation is optimized. For now, we will assume that \(\Delta x\) =
\(\Delta y\). \begin{equation}
\begin{split} 
\frac{u_{i + 1 ,j}^m + u_{i - 1 ,j}^m - 2 u_{i, j}^m}{(\Delta x)^2} + \frac{u_{i,j + 1}^m + u_{i,j - 1}^m - 2 u_{i,j}^m}{(\Delta y)^2} &= \frac{u_{i,j}^{m + 1} - u_{i,j}^{m}}{2 (\Delta t)} \\
\frac{u_{i + 1 ,j}^m + u_{i - 1 ,j}^m - 2 u_{i, j}^m}{(\Delta x)^2} + \frac{u_{i,j + 1}^m + u_{i,j - 1}^m - 2 u_{i,j}^m}{(\Delta x)^2} &= \frac{u_{i,j}^{m + 1} - u_{i,j}^{m}}{2 (\Delta t)} \\
\frac{u_{i + 1 ,j}^m + u_{i - 1 ,j}^m - 2 u_{i, j}^m + u_{i,j + 1}^m + u_{i,j - 1}^m - 2 u_{i,j}^m}{(\Delta x)^2} &= \frac{u_{i,j}^{m + 1} - u_{i,j}^{m - 1}}{2 \Delta t} \\
\frac{u_{i + 1 ,j}^m + u_{i - 1 ,j}^m + u_{i,j + 1}^m + u_{i,j - 1}^m - 4 u_{i,j}^m}{(\Delta x)^2} &= \frac{u_{i,j}^{m + 1} - u_{i,j}^{m}}{2 \Delta t} \\
\frac{(2 \Delta t)}{(\Delta x)^2}(u_{i + 1 ,j}^m + u_{i - 1 ,j}^m + u_{i,j + 1}^m + u_{i,j - 1}^m - 4 u_{i,j}^m) + u_{i,j}^{m} &= u_{i,j}^{m + 1}
\end{split}
\end{equation}

    Let's look at the first few terms. Set \begin{equation}
s = \frac{(2 \Delta t)}{(\Delta x)^2}
\end{equation}

\begin{equation}
\begin{split} 
u_{1,1}^{1} &= s \cdot u_{2,1}^{0} + s \cdot u_{0,1}^{0} + s \cdot u_{1,2}^{0} + s \cdot u_{1,0}^{0} - 4 s \cdot u_{1,1}^{0} + u_{1,1}^{0} \\
&= s \cdot u_{2,1}^{0} + s \cdot u_{0,1}^{0} + s \cdot u_{1,2}^{0} + s \cdot u_{1,0}^{0}  + (1 - 4 s) \cdot u_{1,1}^{0} \\
u_{2,1}^{1} &= s \cdot u_{3,1}^{0} + s \cdot u_{1,1}^{0} + s \cdot u_{2,2}^{0} + s \cdot u_{2,0}^{0}  + (1 - 4 s) \cdot u_{2,1}^{0} \\
u_{1,2}^{1} &= s \cdot u_{2,2}^{0} + s \cdot u_{0,2}^{0} + s \cdot u_{1,3}^{0} + s \cdot u_{1,1}^{0}  + (1 - 4 s) \cdot u_{1,2}^{0} \\
u_{2,2}^{1} &= s \cdot u_{3,2}^{0} + s \cdot u_{1,2}^{0} + s \cdot u_{2,3}^{0} + s \cdot u_{2,1}^{0}  + (1 - 4 s) \cdot u_{2,2}^{0} \\
u_{1,1}^{2} &= s \cdot u_{2,1}^{1} + s \cdot u_{0,1}^{1} + s \cdot u_{1,2}^{1} + s \cdot u_{1,0}^{1}  + (1 - 4 s) \cdot u_{1,1}^{1} \\
u_{2,1}^{2} &= s \cdot u_{3,1}^{1} + s \cdot u_{1,1}^{1} + s \cdot u_{2,2}^{1} + s \cdot u_{2,0}^{1}  + (1 - 4 s) \cdot u_{2,1}^{1} \\
u_{1,2}^{2} &= s \cdot u_{2,2}^{1} + s \cdot u_{0,2}^{1} + s \cdot u_{1,3}^{1} + s \cdot u_{1,1}^{1}  + (1 - 4 s) \cdot u_{1,2}^{1} \\
u_{2,2}^{2} &= s \cdot u_{3,2}^{1} + s \cdot u_{1,2}^{1} + s \cdot u_{2,3}^{1} + s \cdot u_{2,1}^{1}  + (1 - 4 s) \cdot u_{2,2}^{1} \\
u_{1,1}^{3} &= s \cdot u_{2,1}^{2} + s \cdot u_{0,1}^{2} + s \cdot u_{1,2}^{2} + s \cdot u_{1,0}^{2}  + (1 - 4 s) \cdot u_{1,1}^{2} \\
u_{2,1}^{3} &= s \cdot u_{3,1}^{2} + s \cdot u_{1,1}^{2} + s \cdot u_{2,2}^{2} + s \cdot u_{2,0}^{2}  + (1 - 4 s) \cdot u_{2,1}^{2} \\
u_{1,2}^{3} &= s \cdot u_{2,2}^{2} + s \cdot u_{0,2}^{2} + s \cdot u_{1,3}^{2} + s \cdot u_{1,1}^{2}  + (1 - 4 s) \cdot u_{1,2}^{2} \\
u_{2,2}^{3} &= s \cdot u_{3,2}^{2} + s \cdot u_{1,2}^{2} + s \cdot u_{2,3}^{2} + s \cdot u_{2,1}^{2}  + (1 - 4 s) \cdot u_{2,2}^{2} \\
\vdots \\
u_{i,j}^{m + 1} &= s \cdot u_{i + 1 ,j}^m + s \cdot u_{i - 1 ,j}^m + s \cdot u_{i,j + 1}^m + s \cdot u_{i,j - 1}^m + (1 - 4 s) \cdot u_{i,j}^m  
\end{split}
\end{equation}

Rewrite as \begin{equation}
\begin{split} 
u_{i,j}^{m+1} &= s \cdot u_{i-1,j}^m + s \cdot u_{i,j-1}^m + (1-4s) \cdot u_{i,j}^m + s \cdot u_{i+1,j}^m + s \cdot u_{i,j+1}^m 
\end{split}
\end{equation}

where the boundary conditions are defined as \(\forall i\) when j=0 and
j=M and \(\forall j\) when i=0 and i=L.

If we want to use matrix notation

\begin{equation}
u_{i,j}^{m+1} = 
\begin{pmatrix}
u_{i-1,j}^{m+1} \\ u_{i,j-1}^{m+1} \\ u_{i,j}^{m+1} \\ u_{i+1,j}^{m+1} \\ u_{i,j+1}^{m+1} 
\end{pmatrix}
\end{equation}

\begin{equation}
u_{i,j}^m = 
\begin{pmatrix}
u_{i-1,j}^{m} \\ u_{i,j-1}^{m} \\ u_{i,j}^{m} \\ u_{i+1,j}^{m} \\ u_{i,j+1}^{m} 
\end{pmatrix}
\end{equation}

\begin{equation}
A_{m,n} = 
\begin{pmatrix}
(1-4s) & s & s & 0 & 0 & 0 & 0 & 0 \\
s & (1-4s) & s & s & 0 & 0 & 0 & 0 \\
s & s & (1-4s) & s & s & 0 & 0 & 0 \\
0 & s & s & (1-4s) & s & s & 0 & 0 \\
0 & 0 & s & s & (1-4s) & s & s & 0 \\
\vdots & \vdots & \vdots & \vdots & \vdots & \vdots & \vdots & \vdots \\
0 & 0 & 0 & 0 & 0 & s & s & (1-4s) 
\end{pmatrix}
\end{equation}

\begin{equation}
b_{i,j}^m = 
\begin{pmatrix}
u_{0,j}^{m} \\ u_{i,0}^{m} \\ 0 \\ u_{L,j}^{m} \\ u_{i,M}^{m} 
\end{pmatrix}
\end{equation}

    \hypertarget{vonneumann-stability-analysis}{%
\subsection{vonNeumann Stability
Analysis}\label{vonneumann-stability-analysis}}

    Describe vN stability to ensure amplification \textless= 1. Follows
general solution of the Diffusion equation\ldots.sin(x)\ldots{}

Q is the amplification or growth factor.

Let \begin{equation}
\begin{split}
u^{m}_{j, l} &= Q^{m}e^{i(\alpha x + \beta y)} \\
u^{m+1}_{j, l} &= Q^{m+1}e^{i(\alpha x + \beta y)} = Q Q^{m}e^{i(\alpha x + \beta y)} \\
u^{m}_{j+1, l} &= Q^{m}e^{i(\alpha (x + \Delta x) + \beta y)} \\
u^{m}_{j-1, l} &= Q^{m}e^{i(\alpha (x - \Delta x) + \beta y)} \\
u^{m}_{j, l+1} &= Q^{m}e^{i(\alpha x + \beta (y + \Delta y))} \\
u^{m}_{j, l-1} &= Q^{m}e^{i(\alpha x + \beta (y - \Delta y))} 
\end{split}
\end{equation}

The descritized equation \begin{equation}
\begin{split}
u_{j,l}^{m + 1} & = \frac{(2 \Delta t)}{(\Delta x)^2}(u_{j + 1,l}^m + u_{j - 1 ,l}^m + u_{j,l + 1}^m + u_{j,l - 1}^m - 4 u_{j,l}^m) + u_{j, l}^{m} 
\end{split}
\end{equation} becomes \begin{equation}
\begin{split}
Q Q^{m}e^{i(\alpha x + \beta y)} = \frac{(2 \Delta t)}{(\Delta x)^2} \cdot \left(Q^{m}e^{i(\alpha (x + \Delta x) + \beta y)} +  Q^{m}e^{i(\alpha (x - \Delta x) + \beta y)} + Q^{m}e^{i(\alpha x + \beta (y + \Delta y))} + 
Q^{m}e^{i(\alpha x + \beta (y - \Delta y))} - 4 \cdot Q^{m}e^{i(\alpha x + \beta y)}\right) + Q^{m}e^{i(\alpha x + \beta y)} 
\end{split}
\end{equation} Let \begin{equation}
s = \frac{(2 \Delta t)}{(\Delta x)^2}
\end{equation}

Recall \begin{equation}
cos(x) = \frac{e^{ix} + e^{-ix}}{2} 
\end{equation}

Divide both sides by \(Q^{m}e^{i(\alpha x + \beta y)}\) \begin{equation}
\begin{split}
Q &= s \cdot (e^{i(\alpha \Delta x)} + e^{i(\alpha (-\Delta x))} + e^{i(\beta \Delta y)} + e^{i(\beta (-\Delta y))} - 4) + 1 \\
&= s \cdot (e^{i(\alpha \Delta x)} + e^{-i(\alpha \Delta x)} + e^{i(\beta \Delta y)} + e^{-i(\beta \Delta y)} - 4) + 1 \\
&= s \cdot (2 cos (\alpha \Delta x) + 2 cos (\beta \Delta y) - 4) + 1 \\
&= 1 + 2 s \cdot (cos (\alpha \Delta x) + cos (\beta \Delta y) - 2) \\
&= 1 + 2 s \cdot (cos (\alpha \Delta x) + cos (\beta \underbrace{\Delta y}_{\text{=$\Delta$x, earlier assumption}}) - 2) \\
&= 1 + 2 s \cdot (2 cos (\alpha \Delta x) - 2) \\
&= 1 + 4 s \cdot (cos (\alpha \Delta x) - 1)
\end{split}
\end{equation}

Stability exists \(\vert Q \vert \leq\) 1. The goal is to determine the
largest value of s such that \(\vert Q \vert \leq\) 1. We need to
determine when Q is maximized. By inspection, Q is at its largest value
when \(\alpha \Delta x\) = \(\pi\). \begin{equation}
\begin{split}
Q &= 1 + 4 s \cdot (cos (\pi) - 1) \\
&= 1 + 4 s \cdot (-1 - 1) \\
&= 1 + 4 s \cdot (-2) \\
&= 1 - 8 s 
\end{split}
\end{equation}

Since \(\vert Q \vert \leq\) 1, the following condition arises:
\begin{equation}
-1 \leq 1 - 8 s \leq 1
\end{equation}

For 1 - 8 s \(\leq\) 1, we have s \(\leq\) 0. For -1 \(\leq\) 1 - 8 s,
we have s \(\leq \frac{1}{4}\). This means that there is a relationship
between stability factor, s, and the values that can be chosen for
\(\Delta x\) and \(\Delta t\). In other words, the number of time
segments is directly related to the size of the \(\Delta x\):
\begin{equation}
\begin{split}
s \leq \frac{1}{4} = \frac{2 \Delta t}{(\Delta x)^2} \\
\Delta t = \frac{(\Delta x)^2}{8}
\end{split}
\end{equation}

    \hypertarget{numerical-solution-in-cylindrical-coordinates}{%
\subsection{Numerical Solution in Cylindrical
Coordinates}\label{numerical-solution-in-cylindrical-coordinates}}

    Recall the PDE for a 2-dimensional cylindrical surface: \begin{equation}
\begin{split}
\frac {\partial^2 u}{\partial \rho^2} + \frac{1}{\rho} \frac {\partial u}{\partial \rho} + 
\frac{1}{\rho^2} \frac {\partial^2 u}{\partial \phi^2} = \frac {\partial u}{\partial t}
\end{split}
\end{equation}) Centered-space and forward in time. Combinine equations
xx and yy, the finite difference method generates the following terms to
replace the partial derivatives: \begin{equation}
\begin{split} 
\frac{df}{dt} &\approx \frac{f(\rho, \phi, t + \Delta t) - f(\rho, \phi, t)}{2 (\Delta t)} \\
&= \frac{u_{i,j}^{m + 1} - u_{i,j}^{m}}{2 (\Delta t)} \\
\frac{df}{d\rho} &\approx \frac{f(\rho +\Delta \rho, \phi, t) - f(\rho - \Delta \rho, \phi, t )}{2 (\Delta \rho)} \\
&= \frac{u_{i + 1,j}^{m} - u_{i -1 ,j}^{m}}{2 (\Delta \rho)} \\
\frac{d^2f}{d\rho^2} &\approx \frac{f(\rho + \Delta \rho, \phi, t) + f(\rho - \Delta \rho, \phi, t) - 2 f(\rho, \phi, t)}{(\Delta \rho)^2} \\
&= \frac{u_{i + 1 ,j}^m + u_{i - 1 ,j}^m - 2 u_{i,j}^m}{(\Delta \rho)^2} \\
\frac{d^2f}{d\phi^2} &\approx \frac{f(\rho, \phi + \Delta, t) + f(\rho, \phi - \Delta \phi, t) - 2 f(\rho, \phi, t)}{(\Delta \phi)^2} \\
&= \frac{u_{i,j + 1}^m + u_{i,j - 1}^m - 2 u_{i,j}^m}{(\Delta \phi)^2}  
\end{split} 
\end{equation}

Since the coordinate \(\rho\) is discretized, it get replaced by
\(\rho_i\) since we have to us the value of \(\rho\) at each point in
the mesh. \begin{equation}
\begin{split}
\frac {\partial u}{\partial t} &= \frac {\partial^2 u}{\partial \rho^2} + \frac{1}{\rho} \frac {\partial u}{\partial \rho} + 
\frac{1}{\rho^2} \frac {\partial^2 u}{\partial \phi^2} 
\end{split}
\end{equation}

becomes

\begin{equation}
\begin{split}
\frac{u_{i,j}^{m + 1} - u_{i,j}^{m}}{2 (\Delta t)} &= \frac{u_{i + 1 ,j}^m + u_{i - 1 ,j}^m - 2 u_{i,j}^m}{(\Delta \rho)^2} + 
\frac{1}{\rho_i} \frac{u_{i + 1,j}^{m} - u_{i -1 ,j}^{m}}{2 (\Delta \rho)} +
\frac{1}{(\rho_i)^2} \frac{u_{i,j + 1}^m + u_{i,j - 1}^m - 2 u_{i,j}^m}{(\Delta \phi)^2} \\
u_{i,j}^{m + 1} - u_{i,j}^{m} &= 2 (\Delta t) \left[\frac{u_{i + 1 ,j}^m + u_{i - 1 ,j}^m - 2 u_{i,j}^m}{(\Delta \rho)^2} + 
\frac{1}{\rho_i} \frac{u_{i + 1,j}^{m} - u_{i -1 ,j}^{m}}{2 (\Delta \rho)} +
\frac{1}{(\rho_i)^2} \frac{u_{i,j + 1}^m + u_{i,j - 1}^m - 2 u_{i,j}^m}{(\Delta \phi)^2} \right]  \\
u_{i,j}^{m + 1} &= 2 (\Delta t) \left[\frac{u_{i + 1 ,j}^m + u_{i - 1 ,j}^m - 2 u_{i,j}^m}{(\Delta \rho)^2} + 
\frac{1}{\rho_i} \frac{u_{i + 1,j}^{m} - u_{i -1 ,j}^{m}}{2 (\Delta \rho)} +
\frac{1}{(\rho_i)^2} \frac{u_{i,j + 1}^m + u_{i,j - 1}^m - 2 u_{i,j}^m}{(\Delta \phi)^2} \right] +  u_{i,j}^{m} \\
&= 2 (\Delta t) \left[\frac{u_{i + 1 ,j}^m + u_{i - 1 ,j}^m - 2 u_{i,j}^m}{(\Delta \rho)^2} + 
\frac{1}{\Delta \rho \cdot i} \frac{u_{i + 1,j}^{m} - u_{i -1 ,j}^{m}}{2 (\Delta \rho)} +
\frac{1}{(\Delta \rho \cdot i)^2} \frac{u_{i,j + 1}^m + u_{i,j - 1}^m - 2 u_{i,j}^m}{(\Delta \phi)^2} \right] +  u_{i,j}^{m} \\
&= \frac{2 (\Delta t)}{(\Delta \rho)^2} \left[\left (u_{i + 1 ,j}^m + u_{i - 1 ,j}^m - 2 u_{i,j}^m \right) + 
\frac{u_{i + 1,j}^{m} - u_{i -1 ,j}^{m}}{2 \cdot i} +
\frac{u_{i,j + 1}^m + u_{i,j - 1}^m - 2 u_{i,j}^m}{i^2 \cdot (\Delta \phi)^2} \right] +  u_{i,j}^{m} 
\end{split}
\end{equation}

This was solved for \(\rho\) \textgreater{} 0. We need to look at the
case when \(\rho\) = 0 as a special case, because the PDE terms
\begin{equation}
\lim_{\rho^{+} \rightarrow 0} \frac{1}{\rho}\rightarrow \infty 
\end{equation} For completeness, the previous limit specifies
\(\rho^{+} \rightarrow 0\) as approaching 0 from the positive side of
the number line as we do not consider negative length values.

    \hypertarget{special-case-rho-0}{%
\subsubsection{\texorpdfstring{Special Case \(\rho\) =
0}{Special Case \textbackslash rho = 0}}\label{special-case-rho-0}}

    So, how do we discretize the PDE in this case. First, we can use use
L'Hopital's rule to determine if this term converges, and then, to which
value.

\begin{equation}
\lim_{\rho^{+} \rightarrow 0} \frac{1}{\rho} \frac {\partial u}{\partial \rho} \rightarrow \infty \\
L'Hopital's Rule: \\
\lim_{\rho^{+} \rightarrow 0} \frac{\frac {\partial u}{\partial \rho} \left(\frac {\partial u}{\partial \rho} \right) }{\frac {\partial u}{\partial \rho} \left(\rho\right)} =
\lim_{\rho^{+} \rightarrow 0} \frac{\frac{\partial^2 u}{\partial \rho^2}}{1} =
\frac{\partial^2 u}{\partial \rho^2}
\end{equation}

Now, for \$\rho \$ = 0 (i = 0) the resulting discretized PDE is
\begin{equation}
\begin{split}
s &= 2 \frac{\Delta t}{(\Delta \rho)^2} \\
u_{i}^{m + 1} &= s \left(u_{i + 1}^m + u_{i - 1 }^m - 2 u_{i}^m \right) +  u_{i}^{m} \\
u_{0}^{m + 1} &= s \left(u_{1}^m + u_{- 1}^m - 2 u_{0}^m \right) +  u_{0}^{m} \\
&= s \left(u_{1}^m + \underbrace{u_{- 1}^m}_{\text{=0, the term does not exist}} - 2 u_{0}^m \right) +  u_{0}^{m} \\
&= s \left(u_{1}^m - 2 u_{0}^m \right) +  u_{0}^{m} \\
&= s u_{1}^m + (1 - 2s) u_{0}^m 
\end{split}
\end{equation}

Next, we must determine when this solution is numerically stable.

    \hypertarget{von-neumann-stability-analysis-for-the-special-case-rho-0}{%
\subsubsection{\texorpdfstring{Von Neumann Stability Analysis for the
Special Case \(\rho\) =
0}{Von Neumann Stability Analysis for the Special Case \textbackslash rho = 0}}\label{von-neumann-stability-analysis-for-the-special-case-rho-0}}

    As discussed in the rectilinear example, the amplification factor, Q,
must be \(\vert Q \vert \leq\) 1 in order for the numerical solution to
converge (i.e.~stable). Let,

    \begin{equation}
\begin{split}
u^{m}_{0} &= Q^{m}e^{i\alpha \rho} = Q^{m}e^{i\alpha \cdot 0} = Q^{m} \\
u^{m+1}_{0} &= Q^{m+1}e^{i\alpha \rho} = Q Q^{m}e^{i\alpha \rho} = Q Q^{m}e^{i\alpha \cdot 0} = Q Q^{m} \\
u^{m}_{1} &= Q^{m}e^{i\alpha (\rho + \Delta \rho)} = Q^{m}e^{i\alpha (0 + \Delta \rho)} = Q^{m}e^{i\alpha \Delta \rho}
\end{split}
\end{equation}

Then, \begin{equation}
\begin{split}
u_{0}^{m + 1} &= s u_{1}^m + (1 - 2s) u_{0}^m \\
Q Q^{m} &= s \cdot Q^{m}e^{i\alpha \Delta \rho} + (1-2s) Q^{m} \\
Q &= s \cdot e^{i\alpha \Delta \rho} + (1-2s) \\
&= s \cdot \left( cos(\alpha \Delta \rho) + i sin(\alpha \Delta \rho) \right) + (1-2s) \\
&= s \cdot cos(\alpha \Delta \rho) + i \cdot s \cdot sin(\alpha \Delta \rho)  + (1-2s) \\
&= \left( s \cdot cos(\alpha \Delta \rho) + (1-2s) \right)  + i \cdot s \cdot sin(\alpha \Delta \rho)  
\end{split}
\end{equation}

Since \(\vert Q \vert \leq\) 1, the following condition arises:
\begin{equation}
\begin{split}
Q = (s \cdot cos(\alpha \Delta \rho) + (1-2s) ) + i \cdot s \cdot sin(\alpha \Delta \rho)  \\
Q^{*} = (s \cdot cos(\alpha \Delta \rho) + (1-2s) ) - i \cdot s \cdot sin(\alpha \Delta \rho)  
\end{split}
\end{equation}

Therefore, \begin{equation}
\begin{split}
&-1 \leq \sqrt{Q \cdot Q^{*}} \leq 1 \\
&-1 \leq \sqrt{\left[\left( s \cdot cos(\alpha \Delta \rho) + (1-2s) \right)  + i \cdot s \cdot sin(\alpha \Delta \rho)\right] \cdot 
\left[\left( s \cdot cos(\alpha \Delta \rho) + (1-2s) \right)  - i \cdot s \cdot sin(\alpha \Delta \rho)\right]} \leq 1 \\
&-1 \leq \sqrt{\left( s \cdot cos(\alpha \Delta \rho) + (1-2s) \right)^2  + s^2 \cdot sin^2(\alpha \Delta \rho)} \leq 1 \\ 
&-1 \leq \left( s \cdot cos(\alpha \Delta \rho) + (1-2s) \right)^2  + s^2 \cdot sin^2(\alpha \Delta \rho) \leq 1 \\
&-1 \leq s^2 \cdot cos^2(\alpha \Delta \rho) + (1-2s)^2 + 2(1-2s)(s \cdot cos(\alpha \Delta \rho)) + s^2 \cdot sin^2(\alpha \Delta \rho) \leq 1 \\
&-1 \leq \underbrace{s^2 \cdot cos^2(\alpha \Delta \rho) + s^2 \cdot sin^2(\alpha \Delta \rho)}_{\text{=s$^2$, since $ sin^2 x + cos^2x $ = 1}}  + (1-2s)^2 + 2s(1-2s)(cos(\alpha \Delta \rho)) \leq 1 \\
&-1 \leq s^2 + \underbrace{(1-2s)^2}_{=1-4s+4s^2} + 2s(1-2s)(cos(\alpha \Delta \rho)) \leq 1 \\
&-1 \leq 1-4s+5s^2 + 2s(1-2s)(cos(\alpha \Delta \rho)) \leq 1 
\end{split}
\end{equation}

Since \(1-4s+5s^2 + 2s(1-2s)(cos(\alpha \Delta \rho))\), reaches its max
when \(cos(\alpha \Delta \rho)\) = -1 or when
\(\alpha \Delta \rho = \pi\) \begin{equation}
\begin{split}
&-1 \leq 1 - 4s + 5s^2 + 2s(1-2s)(cos(\alpha \Delta \rho)) \leq 1 \\
&-1 \leq 1 - 4s + 5s^2 + 2s(1-2s)\underbrace{(cos(\alpha \Delta \rho))}_{=-1} \leq 1 \\ 
&-1 \leq 1 - 4s + 5s^2 + 2s(1-2s)(-1) \leq 1 \\ 
&-1 \leq 1 - 4s + 5s^2 - 2s(1-2s) \leq 1 \\ 
&-1 \leq 1 - 4s + 5s^2 - 2s + 4s^2 \leq 1 \\ 
&-1 \leq 9s^2 - 6s + 1 \leq 1 
\end{split}
\end{equation}

Taking the right inequality first , \begin{equation}
\begin{split}
9s^2 - 6s + 1 \leq 1  \\
9s^2 - 6s \leq 0  \\
s \cdot (9s - 6) \leq 0  \\
s \leq 0 \\ or, \\
s \leq \frac{2}{3}
\end{split}
\end{equation}

Now, the left inequality , \begin{equation}
\begin{split}
-1 \leq 9s^2 - 6s + 1  \\
0 \leq 9s^2 - 6s + 2 
\end{split}
\end{equation}

The roots of \(9s^2 - 6s + 2\) are complex, and not applicable for this
situation because there are no time steps with complex values - only
real numbers. Therefore, the largest value of s that will satisfy the
vonNewman stability condition is s \(\leq \frac{2}{3}\) for
\(\rho = 0\). We have already accepted a value of s \(\leq \frac{1}{2}\)
for \(\rho \gt\) 0. So, it is reasonable to use s = \(\frac{1}{2}\) for
\(\rho \geq 0\).

    \hypertarget{complete-numerical-solution}{%
\subsubsection{Complete Numerical
Solution}\label{complete-numerical-solution}}

    To summarize, the numerical solution to the heat Equation is \(i \gt 0\)
\begin{equation}
\begin{split}
u_{i,j}^{m + 1} &= 2 (\Delta t) \left[\frac{u_{i + 1 ,j}^m + u_{i - 1 ,j}^m - 2 u_{i,j}^m}{(\Delta \rho)^2} + 
\frac{1}{\rho_i} \frac{u_{i + 1,j}^{m} - u_{i -1 ,j}^{m}}{2 (\Delta \rho)} +
\frac{1}{(\rho_i)^2} \frac{u_{i,j + 1}^m + u_{i,j - 1}^m - 2 u_{i,j}^m}{(\Delta \phi)^2} \right] +  u_{i,j}^{m} 
\end{split}
\end{equation} where \begin{equation}
\begin{split}
s &= 2 \frac{\Delta t}{(\Delta \rho)^2}
\end{split}
\end{equation} and \(i = 0\) \begin{equation}
\begin{split}
u_{0}^{m + 1} &= s u_{1}^m + (1 - 2s) u_{0}^m 
\end{split}
\end{equation}

This means we need to know the value of \(u_{1}^0\) and \(u_{0}^0\) as
part of the initial conditions.

    \begin{tcolorbox}[breakable, size=fbox, boxrule=1pt, pad at break*=1mm,colback=cellbackground, colframe=cellborder]
\prompt{In}{incolor}{ }{\boxspacing}
\begin{Verbatim}[commandchars=\\\{\}]

\end{Verbatim}
\end{tcolorbox}

    \hypertarget{examples-in-two-dimensions}{%
\section{Examples In Two-Dimensions}\label{examples-in-two-dimensions}}

    \hypertarget{example-in-a-square-plane}{%
\subsection{Example In A Square Plane}\label{example-in-a-square-plane}}

    This example illustrates the temperature distribution in a square plane
over time. The Initial condition, that is time = 0, there is a small 10
x 10 area near the center of the square that is 50,000\(^{\circ}\)C. The
boundary conditions are such that along the y-axis at x=0 and x=L, the
temperature is 10,000\(^{\circ}\)C. The number of time slices is 100.

    \begin{tcolorbox}[breakable, size=fbox, boxrule=1pt, pad at break*=1mm,colback=cellbackground, colframe=cellborder]
\prompt{In}{incolor}{1}{\boxspacing}
\begin{Verbatim}[commandchars=\\\{\}]
\PY{c+c1}{\PYZsh{} Filename:  Plot2DHeat.py}
\PY{c+c1}{\PYZsh{} Purpose:   }
\PY{c+c1}{\PYZsh{}}
\PY{o}{\PYZpc{}}\PY{k}{matplotlib} notebook

\PY{c+c1}{\PYZsh{} imports}
\PY{k+kn}{import}      \PY{n+nn}{sys}
\PY{k+kn}{import}      \PY{n+nn}{math} \PY{k}{as} \PY{n+nn}{mth}
\PY{k+kn}{import}      \PY{n+nn}{matplotlib}\PY{n+nn}{.}\PY{n+nn}{pyplot} \PY{k}{as} \PY{n+nn}{plt}
\PY{k+kn}{import}      \PY{n+nn}{matplotlib}\PY{n+nn}{.}\PY{n+nn}{animation} \PY{k}{as} \PY{n+nn}{animation}
\PY{k+kn}{from}        \PY{n+nn}{matplotlib} \PY{k+kn}{import} \PY{n}{cm}
\PY{k+kn}{from}        \PY{n+nn}{matplotlib}\PY{n+nn}{.}\PY{n+nn}{ticker} \PY{k+kn}{import} \PY{n}{LinearLocator}\PY{p}{,} \PY{n}{FormatStrFormatter}
\PY{k+kn}{import}      \PY{n+nn}{matplotlib}
\PY{k+kn}{import}      \PY{n+nn}{plotly} \PY{k}{as} \PY{n+nn}{pl}
\PY{k+kn}{import}      \PY{n+nn}{numpy} \PY{k}{as} \PY{n+nn}{np}
\PY{k+kn}{from}        \PY{n+nn}{mpl\PYZus{}toolkits}\PY{n+nn}{.}\PY{n+nn}{mplot3d} \PY{k+kn}{import} \PY{n}{Axes3D}
\PY{k+kn}{from}        \PY{n+nn}{IPython}\PY{n+nn}{.}\PY{n+nn}{display} \PY{k+kn}{import} \PY{n}{HTML}

\PY{k}{class} \PY{n+nc}{HeatOnRectangle}\PY{p}{(}\PY{p}{)}\PY{p}{:}
    \PY{k}{def} \PY{n+nf+fm}{\PYZus{}\PYZus{}init\PYZus{}\PYZus{}}\PY{p}{(}\PY{n+nb+bp}{self}\PY{p}{,} \PY{n}{xmax}\PY{o}{=} \PY{l+m+mi}{100}\PY{p}{,} \PY{n}{ymax}\PY{o}{=} \PY{l+m+mi}{100}\PY{p}{,} \PY{n}{tmax}\PY{o}{=}\PY{l+m+mi}{10}\PY{p}{,} \PY{n}{dx} \PY{o}{=} \PY{l+m+mf}{0.1}\PY{p}{,} \PY{n}{dy}\PY{o}{=} \PY{l+m+mf}{0.1}\PY{p}{,} \PY{n}{dt}\PY{o}{=}\PY{l+m+mf}{0.01}\PY{p}{,} \PY{n}{frames}\PY{o}{=}\PY{l+m+mi}{50}\PY{p}{,} \PY{n}{bDebug} \PY{o}{=} \PY{k+kc}{False}\PY{p}{)}\PY{p}{:}
        \PY{c+c1}{\PYZsh{}Global debug}
        \PY{n+nb+bp}{self}\PY{o}{.}\PY{n}{bDebug} \PY{o}{=} \PY{n}{bDebug}
        
        \PY{c+c1}{\PYZsh{}Init physical grid for calculation}
        \PY{n+nb+bp}{self}\PY{o}{.}\PY{n}{dt} \PY{o}{=} \PY{n}{dt}
        \PY{n+nb+bp}{self}\PY{o}{.}\PY{n}{dx} \PY{o}{=} \PY{n}{dx}
        \PY{n+nb+bp}{self}\PY{o}{.}\PY{n}{dy} \PY{o}{=} \PY{n}{dy}
        \PY{n+nb+bp}{self}\PY{o}{.}\PY{n}{Tmax} \PY{o}{=} \PY{n}{tmax}
        \PY{n+nb+bp}{self}\PY{o}{.}\PY{n}{Xmax} \PY{o}{=} \PY{n}{xmax}
        \PY{n+nb+bp}{self}\PY{o}{.}\PY{n}{Ymax} \PY{o}{=} \PY{n}{ymax}
        \PY{n+nb+bp}{self}\PY{o}{.}\PY{n}{frames} \PY{o}{=} \PY{n}{frames}
        
        \PY{n+nb+bp}{self}\PY{o}{.}\PY{n}{t} \PY{o}{=} \PY{n}{np}\PY{o}{.}\PY{n}{linspace}\PY{p}{(}\PY{l+m+mi}{0}\PY{p}{,} \PY{n}{tmax}\PY{p}{,} \PY{l+m+mi}{1}\PY{p}{)}
        \PY{n+nb+bp}{self}\PY{o}{.}\PY{n}{x} \PY{o}{=} \PY{n}{np}\PY{o}{.}\PY{n}{linspace}\PY{p}{(}\PY{l+m+mi}{0}\PY{p}{,} \PY{n}{xmax}\PY{p}{,} \PY{n}{xmax}\PY{o}{+}\PY{l+m+mi}{1}\PY{p}{)}
        \PY{n+nb+bp}{self}\PY{o}{.}\PY{n}{y} \PY{o}{=} \PY{n}{np}\PY{o}{.}\PY{n}{linspace}\PY{p}{(}\PY{l+m+mi}{0}\PY{p}{,} \PY{n}{ymax}\PY{p}{,} \PY{n}{ymax}\PY{o}{+}\PY{l+m+mi}{1}\PY{p}{)}   
        \PY{n+nb+bp}{self}\PY{o}{.}\PY{n}{u} \PY{o}{=} \PY{n}{np}\PY{o}{.}\PY{n}{zeros}\PY{p}{(}\PY{p}{[}\PY{n+nb}{len}\PY{p}{(}\PY{n+nb+bp}{self}\PY{o}{.}\PY{n}{x}\PY{p}{)}\PY{p}{,} \PY{n+nb}{len}\PY{p}{(}\PY{n+nb+bp}{self}\PY{o}{.}\PY{n}{y}\PY{p}{)}\PY{p}{]}\PY{p}{,} \PY{n+nb}{float}\PY{p}{)}
        \PY{n+nb+bp}{self}\PY{o}{.}\PY{n}{uAnimation} \PY{o}{=} \PY{n}{np}\PY{o}{.}\PY{n}{zeros}\PY{p}{(}\PY{p}{[}\PY{n+nb}{len}\PY{p}{(}\PY{n+nb+bp}{self}\PY{o}{.}\PY{n}{x}\PY{p}{)}\PY{p}{,} \PY{n+nb}{len}\PY{p}{(}\PY{n+nb+bp}{self}\PY{o}{.}\PY{n}{y}\PY{p}{)}\PY{p}{,} \PY{n+nb+bp}{self}\PY{o}{.}\PY{n}{frames} \PY{o}{+} \PY{l+m+mi}{1}\PY{p}{]}\PY{p}{,} \PY{n+nb}{float}\PY{p}{)}
        
        \PY{c+c1}{\PYZsh{}init graphical output}
        \PY{n+nb+bp}{self}\PY{o}{.}\PY{n}{fig} \PY{o}{=} \PY{n}{plt}\PY{o}{.}\PY{n}{figure}\PY{p}{(}\PY{p}{)}                                 \PY{c+c1}{\PYZsh{} Create figure object}
        \PY{n+nb+bp}{self}\PY{o}{.}\PY{n}{fig}\PY{o}{.}\PY{n}{set\PYZus{}size\PYZus{}inches}\PY{p}{(}\PY{l+m+mi}{10}\PY{p}{,} \PY{l+m+mi}{12}\PY{p}{)}
        
        \PY{n+nb+bp}{self}\PY{o}{.}\PY{n}{ax} \PY{o}{=} \PY{n}{Axes3D}\PY{p}{(}\PY{n+nb+bp}{self}\PY{o}{.}\PY{n}{fig}\PY{p}{,} \PY{n}{auto\PYZus{}add\PYZus{}to\PYZus{}figure}\PY{o}{=}\PY{k+kc}{False}\PY{p}{)}    \PY{c+c1}{\PYZsh{} Create axes object}
        \PY{n+nb+bp}{self}\PY{o}{.}\PY{n}{fig}\PY{o}{.}\PY{n}{add\PYZus{}axes}\PY{p}{(}\PY{n+nb+bp}{self}\PY{o}{.}\PY{n}{ax}\PY{p}{)}
        \PY{n+nb+bp}{self}\PY{o}{.}\PY{n}{surf}\PY{p}{,} \PY{o}{=} \PY{n}{plt}\PY{o}{.}\PY{n}{plot}\PY{p}{(}\PY{p}{[}\PY{p}{]}\PY{p}{,} \PY{p}{[}\PY{p}{]}\PY{p}{,} \PY{p}{[}\PY{p}{]}\PY{p}{)}                       \PY{c+c1}{\PYZsh{} Create empty 3D line object}
        \PY{n+nb+bp}{self}\PY{o}{.}\PY{n}{ax}\PY{o}{.}\PY{n}{set\PYZus{}xlabel}\PY{p}{(}\PY{l+s+s2}{\PYZdq{}}\PY{l+s+s2}{x (cm)}\PY{l+s+s2}{\PYZdq{}}\PY{p}{)}
        \PY{n+nb+bp}{self}\PY{o}{.}\PY{n}{ax}\PY{o}{.}\PY{n}{set\PYZus{}ylabel}\PY{p}{(}\PY{l+s+s2}{\PYZdq{}}\PY{l+s+s2}{y (cm)}\PY{l+s+s2}{\PYZdq{}}\PY{p}{)}
        \PY{n+nb+bp}{self}\PY{o}{.}\PY{n}{ax}\PY{o}{.}\PY{n}{set\PYZus{}zlabel}\PY{p}{(}\PY{l+s+s2}{\PYZdq{}}\PY{l+s+s2}{Temp (C) u(x,y,t)}\PY{l+s+s2}{\PYZdq{}}\PY{p}{)}      
        \PY{n+nb+bp}{self}\PY{o}{.}\PY{n}{X\PYZus{}ax}\PY{p}{,} \PY{n+nb+bp}{self}\PY{o}{.}\PY{n}{Y\PYZus{}ax} \PY{o}{=} \PY{n}{np}\PY{o}{.}\PY{n}{meshgrid}\PY{p}{(}\PY{n+nb+bp}{self}\PY{o}{.}\PY{n}{x}\PY{p}{,}\PY{n+nb+bp}{self}\PY{o}{.}\PY{n}{y}\PY{p}{)}
        \PY{c+c1}{\PYZsh{}self.ax.text(5,10,\PYZsq{}Title\PYZsq{})}
        
    \PY{c+c1}{\PYZsh{} end of \PYZus{}\PYZus{}init\PYZus{}\PYZus{}}
    
    \PY{k}{def} \PY{n+nf+fm}{\PYZus{}\PYZus{}str\PYZus{}\PYZus{}}\PY{p}{(}\PY{n+nb+bp}{self}\PY{p}{)}\PY{p}{:}
        
        \PY{n+nb}{print} \PY{p}{(}\PY{l+s+s2}{\PYZdq{}}\PY{l+s+s2}{u = }\PY{l+s+si}{\PYZpc{}d}\PY{l+s+s2}{\PYZdq{}}\PY{o}{.}\PY{n}{format}\PY{p}{(}\PY{n}{u}\PY{p}{)}\PY{p}{)}    
    \PY{c+c1}{\PYZsh{} end of \PYZus{}\PYZus{}str\PYZus{}\PYZus{}}

    \PY{k}{def} \PY{n+nf}{initSurface}\PY{p}{(}\PY{n+nb+bp}{self}\PY{p}{)}\PY{p}{:}
        \PY{l+s+sd}{\PYZsq{}\PYZsq{}\PYZsq{}}
\PY{l+s+sd}{            x and y are lists (vectors) that define the length and width of the rectangle}
\PY{l+s+sd}{            }
\PY{l+s+sd}{            return an array initialized with zero\PYZsq{}s}
\PY{l+s+sd}{        \PYZsq{}\PYZsq{}\PYZsq{}}
        \PY{k}{if} \PY{p}{(}\PY{n+nb+bp}{self}\PY{o}{.}\PY{n}{bDebug} \PY{o}{==} \PY{k+kc}{True}\PY{p}{)}\PY{p}{:}
            \PY{n+nb}{print}\PY{p}{(}\PY{l+s+s2}{\PYZdq{}}\PY{l+s+s2}{In initSurface}\PY{l+s+s2}{\PYZdq{}}\PY{p}{)}
            
        \PY{n+nb+bp}{self}\PY{o}{.}\PY{n}{u} \PY{o}{=} \PY{n}{np}\PY{o}{.}\PY{n}{zeros}\PY{p}{(}\PY{p}{[}\PY{n+nb}{len}\PY{p}{(}\PY{n+nb+bp}{self}\PY{o}{.}\PY{n}{x}\PY{p}{)}\PY{p}{,} \PY{n+nb}{len}\PY{p}{(}\PY{n+nb+bp}{self}\PY{o}{.}\PY{n}{y}\PY{p}{)}\PY{p}{]}\PY{p}{,} \PY{n+nb}{float}\PY{p}{)}

        \PY{k}{return} 

    \PY{k}{def} \PY{n+nf}{applyInitialConditionsAnim}\PY{p}{(}\PY{n+nb+bp}{self}\PY{p}{)}\PY{p}{:}
    
        \PY{k}{if} \PY{p}{(}\PY{n+nb+bp}{self}\PY{o}{.}\PY{n}{bDebug} \PY{o}{==} \PY{k+kc}{True}\PY{p}{)}\PY{p}{:}
            \PY{n+nb}{print}\PY{p}{(}\PY{l+s+s2}{\PYZdq{}}\PY{l+s+s2}{applyInitialConditionsAnim}\PY{l+s+s2}{\PYZdq{}}\PY{p}{)}
        
        \PY{c+c1}{\PYZsh{}find some grid element near the center}
        \PY{n}{a} \PY{o}{=} \PY{n}{np}\PY{o}{.}\PY{n}{int32}\PY{p}{(}\PY{n}{np}\PY{o}{.}\PY{n}{floor}\PY{p}{(}\PY{n+nb}{len}\PY{p}{(}\PY{n+nb+bp}{self}\PY{o}{.}\PY{n}{x}\PY{p}{)}\PY{o}{/}\PY{l+m+mi}{2}\PY{p}{)}\PY{p}{)}
        \PY{n}{b} \PY{o}{=} \PY{n}{np}\PY{o}{.}\PY{n}{int32}\PY{p}{(}\PY{n}{np}\PY{o}{.}\PY{n}{floor}\PY{p}{(}\PY{n+nb}{len}\PY{p}{(}\PY{n+nb+bp}{self}\PY{o}{.}\PY{n}{y}\PY{p}{)}\PY{o}{/}\PY{l+m+mi}{2}\PY{p}{)}\PY{p}{)}
                
        \PY{n+nb+bp}{self}\PY{o}{.}\PY{n}{uAnimation}\PY{p}{[}\PY{n}{a}\PY{o}{\PYZhy{}}\PY{l+m+mi}{10}\PY{p}{:}\PY{n}{a}\PY{o}{+}\PY{l+m+mi}{10}\PY{p}{,} \PY{n}{b}\PY{o}{\PYZhy{}}\PY{l+m+mi}{5}\PY{p}{:}\PY{n}{b}\PY{o}{+}\PY{l+m+mi}{20}\PY{p}{,} \PY{l+m+mi}{0}\PY{p}{]} \PY{o}{=} \PY{l+m+mf}{50000.0}

        \PY{k}{if} \PY{p}{(}\PY{n+nb+bp}{self}\PY{o}{.}\PY{n}{bDebug} \PY{o}{==} \PY{k+kc}{True}\PY{p}{)}\PY{p}{:}
            \PY{n+nb}{print} \PY{p}{(}\PY{n+nb+bp}{self}\PY{o}{.}\PY{n}{uAnimation}\PY{p}{[}\PY{n}{a}\PY{p}{,} \PY{n}{b}\PY{p}{,} \PY{l+m+mi}{0}\PY{p}{]}\PY{p}{)}
            \PY{n+nb}{print} \PY{p}{(}\PY{n+nb+bp}{self}\PY{o}{.}\PY{n}{uAnimation}\PY{p}{)}
        
        \PY{k}{return} 

    \PY{k}{def} \PY{n+nf}{applyBoundaryConditionsAnim}\PY{p}{(}\PY{n+nb+bp}{self}\PY{p}{,} \PY{n}{t}\PY{p}{)}\PY{p}{:}
        \PY{c+c1}{\PYZsh{}along the x\PYZhy{}axis}
        \PY{n+nb+bp}{self}\PY{o}{.}\PY{n}{uAnimation}\PY{p}{[}\PY{p}{:}\PY{p}{,} \PY{l+m+mi}{0}\PY{p}{,} \PY{n}{t}\PY{p}{]} \PY{o}{=} \PY{l+m+mf}{0.0}
        
        \PY{c+c1}{\PYZsh{}along the x\PYZhy{}axis on the upper\PYZhy{}side of the rectangle}
        \PY{n+nb+bp}{self}\PY{o}{.}\PY{n}{uAnimation}\PY{p}{[}\PY{p}{:}\PY{p}{,} \PY{n+nb}{len}\PY{p}{(}\PY{n+nb+bp}{self}\PY{o}{.}\PY{n}{y}\PY{p}{)}\PY{o}{\PYZhy{}}\PY{l+m+mi}{1}\PY{p}{,} \PY{n}{t}\PY{p}{]} \PY{o}{=} \PY{l+m+mf}{0.0}
        
        \PY{c+c1}{\PYZsh{}along the y\PYZhy{}axis}
        \PY{n+nb+bp}{self}\PY{o}{.}\PY{n}{uAnimation}\PY{p}{[}\PY{l+m+mi}{0}\PY{p}{,}\PY{p}{:}\PY{p}{,}\PY{n}{t}\PY{p}{]} \PY{o}{=} \PY{l+m+mf}{10000.0}
        
        \PY{c+c1}{\PYZsh{}along the y\PYZhy{}axis on the right\PYZhy{}side of the rectangle}
        \PY{n+nb+bp}{self}\PY{o}{.}\PY{n}{uAnimation}\PY{p}{[}\PY{n+nb}{len}\PY{p}{(}\PY{n+nb+bp}{self}\PY{o}{.}\PY{n}{x}\PY{p}{)}\PY{o}{\PYZhy{}}\PY{l+m+mi}{1}\PY{p}{,} \PY{p}{:}\PY{p}{,} \PY{n}{t}\PY{p}{]} \PY{o}{=} \PY{l+m+mf}{10000.0}

        \PY{k}{if} \PY{p}{(}\PY{n+nb+bp}{self}\PY{o}{.}\PY{n}{bDebug} \PY{o}{==} \PY{k+kc}{True}\PY{p}{)}\PY{p}{:}
            \PY{n+nb}{print} \PY{p}{(}\PY{n+nb+bp}{self}\PY{o}{.}\PY{n}{uAnimation}\PY{p}{)}
        
        \PY{k}{return}
        
    \PY{k}{def} \PY{n+nf}{calcTempNextTimeSegmentAnim}\PY{p}{(}\PY{n+nb+bp}{self}\PY{p}{,} \PY{n}{t}\PY{p}{)}\PY{p}{:}
        \PY{k}{for} \PY{n}{j} \PY{o+ow}{in} \PY{n+nb}{range}\PY{p}{(}\PY{l+m+mi}{1}\PY{p}{,} \PY{n+nb}{len}\PY{p}{(}\PY{n+nb+bp}{self}\PY{o}{.}\PY{n}{x}\PY{p}{)}\PY{o}{\PYZhy{}}\PY{l+m+mi}{1}\PY{p}{)}\PY{p}{:}
            \PY{k}{if} \PY{p}{(}\PY{n+nb+bp}{self}\PY{o}{.}\PY{n}{bDebug} \PY{o}{==} \PY{k+kc}{True}\PY{p}{)}\PY{p}{:}
                \PY{n+nb}{print} \PY{p}{(}\PY{l+s+s2}{\PYZdq{}}\PY{l+s+s2}{j = }\PY{l+s+s2}{\PYZdq{}}\PY{p}{,} \PY{n}{j}\PY{p}{)}
            \PY{k}{for} \PY{n}{l} \PY{o+ow}{in}  \PY{n+nb}{range}\PY{p}{(}\PY{l+m+mi}{1}\PY{p}{,} \PY{n+nb}{len}\PY{p}{(}\PY{n+nb+bp}{self}\PY{o}{.}\PY{n}{y}\PY{p}{)}\PY{o}{\PYZhy{}}\PY{l+m+mi}{1}\PY{p}{)}\PY{p}{:}
                \PY{k}{if} \PY{p}{(}\PY{n+nb+bp}{self}\PY{o}{.}\PY{n}{bDebug} \PY{o}{==} \PY{k+kc}{True}\PY{p}{)}\PY{p}{:}
                    \PY{n+nb}{print} \PY{p}{(}\PY{l+s+s2}{\PYZdq{}}\PY{l+s+s2}{l = }\PY{l+s+s2}{\PYZdq{}}\PY{p}{,} \PY{n}{l}\PY{p}{)}
                \PY{n+nb+bp}{self}\PY{o}{.}\PY{n}{uAnimation}\PY{p}{[}\PY{n}{j}\PY{p}{,}\PY{n}{l}\PY{p}{,}\PY{n}{t}\PY{p}{]} \PY{o}{=} \PY{p}{(}\PY{l+m+mf}{1.0} \PY{o}{/} \PY{l+m+mf}{4.0}\PY{p}{)} \PY{o}{*} \PY{p}{(}\PY{n+nb+bp}{self}\PY{o}{.}\PY{n}{uAnimation}\PY{p}{[}\PY{n}{j}\PY{o}{+}\PY{l+m+mi}{1}\PY{p}{,}\PY{n}{l}\PY{p}{,}\PY{n}{t}\PY{o}{\PYZhy{}}\PY{l+m+mi}{1}\PY{p}{]} \PY{o}{+} 
                                            \PY{n+nb+bp}{self}\PY{o}{.}\PY{n}{uAnimation}\PY{p}{[}\PY{n}{j}\PY{o}{\PYZhy{}}\PY{l+m+mi}{1}\PY{p}{,}\PY{n}{l}\PY{p}{,}\PY{n}{t}\PY{o}{\PYZhy{}}\PY{l+m+mi}{1}\PY{p}{]} \PY{o}{+} 
                                            \PY{n+nb+bp}{self}\PY{o}{.}\PY{n}{uAnimation}\PY{p}{[}\PY{n}{j}\PY{p}{,}\PY{n}{l}\PY{o}{+}\PY{l+m+mi}{1}\PY{p}{,}\PY{n}{t}\PY{o}{\PYZhy{}}\PY{l+m+mi}{1}\PY{p}{]} \PY{o}{+} 
                                            \PY{n+nb+bp}{self}\PY{o}{.}\PY{n}{uAnimation}\PY{p}{[}\PY{n}{j}\PY{p}{,}\PY{n}{l}\PY{o}{\PYZhy{}}\PY{l+m+mi}{1}\PY{p}{,}\PY{n}{t}\PY{o}{\PYZhy{}}\PY{l+m+mi}{1}\PY{p}{]}\PY{p}{)} 

            \PY{k}{if} \PY{p}{(}\PY{n+nb+bp}{self}\PY{o}{.}\PY{n}{bDebug} \PY{o}{==} \PY{k+kc}{True}\PY{p}{)}\PY{p}{:}
                \PY{n+nb}{print} \PY{p}{(}\PY{n+nb+bp}{self}\PY{o}{.}\PY{n}{uAnimation}\PY{p}{)}

        \PY{k}{return}

    \PY{k}{def} \PY{n+nf}{calcDataAnimation}\PY{p}{(}\PY{n+nb+bp}{self}\PY{p}{,} \PY{n}{tMax}\PY{p}{)}\PY{p}{:}

        \PY{c+c1}{\PYZsh{}Generate the data throughout the time interval}
        \PY{k}{for} \PY{n}{t} \PY{o+ow}{in} \PY{n+nb}{range}\PY{p}{(}\PY{l+m+mi}{1}\PY{p}{,} \PY{n}{tMax}\PY{p}{)}\PY{p}{:}        
            \PY{n+nb+bp}{self}\PY{o}{.}\PY{n}{applyBoundaryConditionsAnim}\PY{p}{(}\PY{n}{t}\PY{p}{)}
            \PY{n+nb+bp}{self}\PY{o}{.}\PY{n}{calcTempNextTimeSegmentAnim}\PY{p}{(}\PY{n}{t}\PY{p}{)}
        
        \PY{k}{return} 
    
    \PY{k}{def} \PY{n+nf}{animatePlot}\PY{p}{(}\PY{n+nb+bp}{self}\PY{p}{,} \PY{n}{nFrame}\PY{p}{,} \PY{n}{plot}\PY{p}{)}\PY{p}{:}

        
        \PY{n+nb+bp}{self}\PY{o}{.}\PY{n}{applyBoundaryConditionsAnim}\PY{p}{(}\PY{n}{nFrame}\PY{p}{)}
        \PY{n+nb+bp}{self}\PY{o}{.}\PY{n}{calcTempNextTimeSegmentAnim}\PY{p}{(}\PY{n}{nFrame}\PY{p}{)}
        
        \PY{n+nb+bp}{self}\PY{o}{.}\PY{n}{ax}\PY{o}{.}\PY{n}{clear}\PY{p}{(}\PY{p}{)}
        \PY{n}{plot} \PY{o}{=} \PY{n+nb+bp}{self}\PY{o}{.}\PY{n}{ax}\PY{o}{.}\PY{n}{plot\PYZus{}surface}\PY{p}{(}\PY{n}{X}\PY{p}{,} \PY{n}{Y}\PY{p}{,} \PY{n+nb+bp}{self}\PY{o}{.}\PY{n}{uAnimation}\PY{p}{[}\PY{p}{:}\PY{p}{,}\PY{p}{:}\PY{p}{,}\PY{n}{nFrame}\PY{p}{]}\PY{p}{,} \PY{n}{cmap}\PY{o}{=}\PY{n}{cm}\PY{o}{.}\PY{n}{coolwarm}\PY{p}{,} \PY{n}{linewidth}\PY{o}{=}\PY{l+m+mi}{0}\PY{p}{,} \PY{n}{antialiased}\PY{o}{=}\PY{k+kc}{False}\PY{p}{)}
        \PY{n}{plt}\PY{o}{.}\PY{n}{title}\PY{p}{(}\PY{l+s+s2}{\PYZdq{}}\PY{l+s+s2}{Heat distribution on a plane}\PY{l+s+s2}{\PYZdq{}}\PY{p}{)}
        
        \PY{k}{return} \PY{n}{plot}\PY{p}{,}
    
    \PY{k}{def} \PY{n+nf}{calcDataForAnimation}\PY{p}{(}\PY{n+nb+bp}{self}\PY{p}{,} \PY{n}{nFrame}\PY{p}{,} \PY{n}{X}\PY{p}{,} \PY{n}{Y}\PY{p}{,} \PY{n}{plot}\PY{p}{)}\PY{p}{:}
    
        \PY{n}{plot}\PY{p}{[}\PY{l+m+mi}{0}\PY{p}{]}\PY{o}{.}\PY{n}{remove}\PY{p}{(}\PY{p}{)}

        \PY{n}{plot}\PY{p}{[}\PY{l+m+mi}{0}\PY{p}{]} \PY{o}{=} \PY{n+nb+bp}{self}\PY{o}{.}\PY{n}{ax}\PY{o}{.}\PY{n}{plot\PYZus{}surface}\PY{p}{(}\PY{n}{X}\PY{p}{,} \PY{n}{Y}\PY{p}{,} \PY{n+nb+bp}{self}\PY{o}{.}\PY{n}{uAnimation}\PY{p}{[}\PY{p}{:}\PY{p}{,}\PY{p}{:}\PY{p}{,}\PY{n}{nFrame}\PY{p}{]}\PY{p}{,} \PY{n}{cmap}\PY{o}{=}\PY{n}{cm}\PY{o}{.}\PY{n}{coolwarm}\PY{p}{,} \PY{n}{linewidth}\PY{o}{=}\PY{l+m+mi}{0}\PY{p}{,} \PY{n}{antialiased}\PY{o}{=}\PY{k+kc}{False}\PY{p}{)}

        \PY{c+c1}{\PYZsh{}self.title.set\PYZus{}text(\PYZdq{}Heat distribution on the plane for time segment \PYZob{}\PYZcb{}\PYZdq{}.format(nFrame))}
        
        \PY{k}{return} 

    \PY{k}{def} \PY{n+nf}{animateSurfacePlot}\PY{p}{(}\PY{n+nb+bp}{self}\PY{p}{,} \PY{n}{T}\PY{p}{,} \PY{n}{X}\PY{p}{,} \PY{n}{Y}\PY{p}{,} \PY{n}{plot}\PY{p}{)}\PY{p}{:}
        \PY{c+c1}{\PYZsh{} call the animator.  blit=True means only re\PYZhy{}draw the parts that have changed.}
        \PY{n}{anim} \PY{o}{=} \PY{n}{animation}\PY{o}{.}\PY{n}{FuncAnimation}\PY{p}{(}\PY{n+nb+bp}{self}\PY{o}{.}\PY{n}{fig}\PY{p}{,} \PY{n+nb+bp}{self}\PY{o}{.}\PY{n}{calcDataForAnimation}\PY{p}{,} \PY{n}{T}\PY{p}{,} \PY{n}{fargs}\PY{o}{=}\PY{p}{(}\PY{n}{X}\PY{p}{,} \PY{n}{Y}\PY{p}{,} \PY{n}{plot}\PY{p}{)}\PY{p}{,} \PY{n}{blit}\PY{o}{=}\PY{k+kc}{True}\PY{p}{)}

        \PY{c+c1}{\PYZsh{} save the animation as an mp4.  This requires ffmpeg or mencoder to be}
        \PY{c+c1}{\PYZsh{} installed.  The extra\PYZus{}args ensure that the x264 codec is used, so that}
        \PY{c+c1}{\PYZsh{} the video can be embedded in html5.  You may need to adjust this for}
        \PY{c+c1}{\PYZsh{} your system: for more information, see}
        \PY{c+c1}{\PYZsh{} http://matplotlib.sourceforge.net/api/animation\PYZus{}api.html}
        \PY{c+c1}{\PYZsh{}anim.save(\PYZsq{}basic\PYZus{}animation.mp4\PYZsq{}, fps=30)}

        \PY{n}{html} \PY{o}{=} \PY{n}{HTML}\PY{p}{(}\PY{n}{anim}\PY{o}{.}\PY{n}{to\PYZus{}html5\PYZus{}video}\PY{p}{(}\PY{p}{)}\PY{p}{)}
        \PY{n}{display}\PY{p}{(}\PY{n}{html}\PY{p}{)}
        \PY{n}{plt}\PY{o}{.}\PY{n}{close}\PY{p}{(}\PY{p}{)}
        
        \PY{k}{return}
    
    \PY{k}{def} \PY{n+nf}{plotSurface}\PY{p}{(}\PY{n+nb+bp}{self}\PY{p}{)}\PY{p}{:}

        \PY{n}{X}\PY{p}{,} \PY{n}{Y} \PY{o}{=} \PY{n}{np}\PY{o}{.}\PY{n}{meshgrid}\PY{p}{(}\PY{n+nb+bp}{self}\PY{o}{.}\PY{n}{x}\PY{p}{,}\PY{n+nb+bp}{self}\PY{o}{.}\PY{n}{y}\PY{p}{)}

        \PY{c+c1}{\PYZsh{} Plot the surface.}
        \PY{n}{plot} \PY{o}{=} \PY{p}{[}\PY{n+nb+bp}{self}\PY{o}{.}\PY{n}{ax}\PY{o}{.}\PY{n}{plot\PYZus{}surface}\PY{p}{(}\PY{n+nb+bp}{self}\PY{o}{.}\PY{n}{X\PYZus{}ax}\PY{p}{,} \PY{n+nb+bp}{self}\PY{o}{.}\PY{n}{Y\PYZus{}ax}\PY{p}{,} \PY{n+nb+bp}{self}\PY{o}{.}\PY{n}{uAnimation}\PY{p}{[}\PY{p}{:}\PY{p}{,}\PY{p}{:}\PY{p}{,}\PY{l+m+mi}{0}\PY{p}{]}\PY{p}{,} \PY{n}{cmap}\PY{o}{=}\PY{n}{cm}\PY{o}{.}\PY{n}{coolwarm}\PY{p}{,} \PY{n}{linewidth}\PY{o}{=}\PY{l+m+mi}{0}\PY{p}{,} \PY{n}{antialiased}\PY{o}{=}\PY{k+kc}{False}\PY{p}{)}\PY{p}{]}

        \PY{c+c1}{\PYZsh{} Add a color bar which maps values to colors.}
        \PY{n}{cb} \PY{o}{=} \PY{n+nb+bp}{self}\PY{o}{.}\PY{n}{fig}\PY{o}{.}\PY{n}{colorbar}\PY{p}{(}\PY{n}{plot}\PY{p}{[}\PY{l+m+mi}{0}\PY{p}{]}\PY{p}{,} \PY{n}{shrink}\PY{o}{=}\PY{l+m+mf}{0.5}\PY{p}{,} \PY{n}{aspect}\PY{o}{=}\PY{l+m+mi}{5}\PY{p}{)}
        \PY{n}{cb}\PY{o}{.}\PY{n}{set\PYZus{}label}\PY{p}{(}\PY{l+s+s2}{\PYZdq{}}\PY{l+s+s2}{Temp (C)}\PY{l+s+s2}{\PYZdq{}}\PY{p}{)}
        
        \PY{k}{return} \PY{n}{X}\PY{p}{,} \PY{n}{Y}\PY{p}{,} \PY{n}{plot}\PY{p}{,}

\PY{c+c1}{\PYZsh{} end of class HeatOnRectangle}

 
\PY{c+c1}{\PYZsh{} *****}
\PY{c+c1}{\PYZsh{} Python entry point}
\PY{c+c1}{\PYZsh{} *****}
\PY{k}{if} \PY{n+nv+vm}{\PYZus{}\PYZus{}name\PYZus{}\PYZus{}} \PY{o}{==} \PY{l+s+s2}{\PYZdq{}}\PY{l+s+s2}{\PYZus{}\PYZus{}main\PYZus{}\PYZus{}}\PY{l+s+s2}{\PYZdq{}}\PY{p}{:}
    \PY{l+s+sd}{\PYZsq{}\PYZsq{}\PYZsq{}}
\PY{l+s+sd}{    Process to create an animiated graphic using FuncAnimation (from http://www.acme.byu.edu/wp\PYZhy{}content/uploads/2018/09/Animation.pdf)}
\PY{l+s+sd}{    1. Compute all data to be plotted.}
\PY{l+s+sd}{    2. Explicitly define figure object.}
\PY{l+s+sd}{    3. Define line objects to be altered dynamically.}
\PY{l+s+sd}{    4. Create function to update line objects.}
\PY{l+s+sd}{    5. Create FuncAnimation object.}
\PY{l+s+sd}{    6. Display using plt.show().}

\PY{l+s+sd}{    Approach from the following sources:}
\PY{l+s+sd}{    https://stackoverflow.com/questions/45712099/updating\PYZhy{}z\PYZhy{}data\PYZhy{}on\PYZhy{}a\PYZhy{}surface\PYZhy{}plot\PYZhy{}in\PYZhy{}matplotlib\PYZhy{}animation}
\PY{l+s+sd}{    https://pythonmatplotlibtips.blogspot.com/2018/11/animation\PYZhy{}3d\PYZhy{}surface\PYZhy{}plot\PYZhy{}artistanimation\PYZhy{}matplotlib.html}
\PY{l+s+sd}{    \PYZsq{}\PYZsq{}\PYZsq{}}
    \PY{n}{s} \PY{o}{=} \PY{o}{\PYZhy{}}\PY{l+m+mf}{0.20}
    \PY{n}{T} \PY{o}{=} \PY{l+m+mi}{100}
    \PY{n}{N} \PY{o}{=} \PY{l+m+mi}{500}
    \PY{n}{L} \PY{o}{=} \PY{l+m+mi}{500}
    \PY{n}{Nx} \PY{o}{=} \PY{l+m+mi}{10000}
    \PY{n}{Ny} \PY{o}{=} \PY{l+m+mi}{10000}
    \PY{n}{Nt} \PY{o}{=} \PY{l+m+mi}{100}
    \PY{n}{dx} \PY{o}{=} \PY{n}{N}\PY{o}{/}\PY{n}{Nx}
    \PY{n}{dy} \PY{o}{=} \PY{n}{L}\PY{o}{/}\PY{n}{Ny}    
    \PY{n}{dt} \PY{o}{=} \PY{n}{mth}\PY{o}{.}\PY{n}{ceil}\PY{p}{(}\PY{l+m+mf}{0.25} \PY{o}{*} \PY{p}{(}\PY{n}{dx}\PY{o}{*}\PY{o}{*}\PY{l+m+mi}{2}\PY{p}{)} \PY{o}{/} \PY{l+m+mi}{2}\PY{p}{)}
    \PY{n}{debug} \PY{o}{=} \PY{k+kc}{False}
    \PY{n}{matplotlib}\PY{o}{.}\PY{n}{matplotlib\PYZus{}fname}\PY{p}{(}\PY{p}{)}

    \PY{c+c1}{\PYZsh{}print (\PYZdq{}Num time segs: \PYZdq{}, dt)}
    \PY{c+c1}{\PYZsh{}sys.exit(\PYZdq{}done\PYZdq{})}

    \PY{c+c1}{\PYZsh{}1. Compute all data to be plotted.}
    \PY{c+c1}{\PYZsh{}2. Explicitly define figure object.}
    \PY{n}{r} \PY{o}{=} \PY{n}{HeatOnRectangle}\PY{p}{(}\PY{n}{N}\PY{p}{,} \PY{n}{L}\PY{p}{,} \PY{n}{T}\PY{p}{,} \PY{n}{dx}\PY{p}{,} \PY{n}{dy}\PY{p}{,} \PY{n}{dt}\PY{p}{,} \PY{n}{T}\PY{p}{,} \PY{n}{debug}\PY{p}{)}
    \PY{n}{r}\PY{o}{.}\PY{n}{applyInitialConditionsAnim}\PY{p}{(}\PY{p}{)}
    \PY{n}{r}\PY{o}{.}\PY{n}{calcDataAnimation}\PY{p}{(}\PY{n}{T}\PY{p}{)}

    \PY{n}{X}\PY{p}{,} \PY{n}{Y}\PY{p}{,} \PY{n}{plot} \PY{o}{=} \PY{n}{r}\PY{o}{.}\PY{n}{plotSurface}\PY{p}{(}\PY{p}{)}
    \PY{n}{r}\PY{o}{.}\PY{n}{animateSurfacePlot}\PY{p}{(}\PY{n}{T}\PY{p}{,} \PY{n}{X}\PY{p}{,} \PY{n}{Y}\PY{p}{,} \PY{n}{plot} \PY{p}{)}
    
    \PY{n+nb}{print} \PY{p}{(}\PY{l+s+s2}{\PYZdq{}}\PY{l+s+s2}{Done!}\PY{l+s+s2}{\PYZdq{}}\PY{p}{)}    
\end{Verbatim}
\end{tcolorbox}

    
    \begin{Verbatim}[commandchars=\\\{\}]
<IPython.core.display.Javascript object>
    \end{Verbatim}

    
    
    \begin{Verbatim}[commandchars=\\\{\}]
<IPython.core.display.HTML object>
    \end{Verbatim}

    
    
    \begin{Verbatim}[commandchars=\\\{\}]
<IPython.core.display.HTML object>
    \end{Verbatim}

    
    \begin{Verbatim}[commandchars=\\\{\}]
Done!
    \end{Verbatim}

    \hypertarget{references}{%
\section{References}\label{references}}

    \hypertarget{derive-vecnabla-urho-phi-and-nabla2-urho-phi}{%
\subsection{\texorpdfstring{Derive \(\vec{\nabla u}(\rho, \phi)\) and
\(\nabla^2 u(\rho, \phi)\)}{Derive \textbackslash vec\{\textbackslash nabla u\}(\textbackslash rho, \textbackslash phi) and \textbackslash nabla\^{}2 u(\textbackslash rho, \textbackslash phi)}}\label{derive-vecnabla-urho-phi-and-nabla2-urho-phi}}

    The process to derive the gradient (in two dimensions), which is a
vector, \begin{equation}
\vec{\nabla u}(\rho,\phi)
\label{eq:eq1} \tag{1}
\end{equation}

and the LaPlacian \begin{equation}
\nabla^2 u(\rho,\phi) = \vec{\nabla u}(\rho,\phi) \cdot \vec{\nabla u}(\rho,\phi)
\label{eq:eq2} \tag{2}
\end{equation}

is to first start with those operators in rectilinear coordinates and
use the the well-known conversion identities: \begin{equation}
x = \rho \cdot \cos(\phi)
\label{eq:eq3a} \tag{3a}
\end{equation}

\begin{equation}
y = \rho \cdot \sin(\phi)
\label{eq:eq3b} \tag{3b}
\end{equation}

or \begin{equation}
\rho = \sqrt{x^2 + y^2}
\label{eq:eq3c} \tag{3c}
\end{equation}

\begin{equation}
\phi = \tan^{-1} \left( \frac{y}{x} \right)
\label{eq:eq3d} \tag{3d}
\end{equation}

and

\begin{equation}
\vec{\mathbf{\nabla}} u(x,y) = \frac{\partial u}{\partial x} \hat{\mathbf{x}} + \frac{\partial u}{\partial y} \hat{\mathbf{y}}
\label{eq:eq4} \tag{4}
\end{equation}

The next thing to consider, since we are dealing with vectors, is how do
the cyclindrical coordingate unit vectors \(\hat{\rho}\), \(\hat{\phi}\)
and \(\hat{z}\) project onto the rectilinear coordinate \(\hat{x}\),
\(\hat{y}\), and \(\hat{z}\). For the purposes of 2 dimensions, we will
only consider the \(\hat{\rho}\) and \(\hat{\phi}\) projection onto the
\(\hat{x}\), \(\hat{y}\) unit vectors. The key to this discussion is
that unlike \(\hat{x}\) and \(\hat{y}\), \(\hat{\rho}\) and
\(\hat{\phi}\) are not constant unit vectors; they vary with location.
Having said that, they are still orthogonal so that
\[\hat{\rho} \cdot \hat{\rho} = 1\] \[\hat{\phi} \cdot \hat{\phi} = 1\]
\[\hat{\rho} \cdot \hat{\phi} = 0\]

\(\hat{\rho}\) and \(\hat{\phi}\) will have components along both the
\(\hat{x}\) and \(\hat{y}\) axes. As such the following is constructed
\[\hat{\rho} = \cos(\phi) \hat{x} + \sin(\phi) \hat{y}\],
\[\hat{\phi} = -\sin(\phi) \hat{x} + \cos(\phi) \hat{y}\]

Then the derivates are:
\[\frac{\partial \hat{\rho}}{\partial \phi} = -\sin(\phi) \hat{x} + \cos(\phi) \hat{y} = \hat{\phi}\]
\[\frac{\partial \hat{\phi}}{\partial \phi} = -\cos(\phi) \hat{x} - \sin(\phi) \hat{y} = -\hat{\rho}\]

In order to calculate the gradient in cyclindral coordinates based on
\(\nabla\) operator in rectilinear coordinates, the previous equations
will need to be changed in terms of \(\hat{x}\) and \(\hat{y}\). This
can be done using matrx algebra.

\[\left(\begin{array}{cc} \cos(\phi) && \sin(\phi) \\ -\sin(\phi) && \cos(\phi) \end{array} \right) 
\left( \begin{array}{c} \hat{x} \\ \hat{y} \end{array} \right) = \left( \begin{array}{c} \hat{\rho} \\ \hat{\phi} \end{array} \right)\]

The determinant of the matrix is\\
\[det = \cos(\phi) \cdot \cos(\phi) - (\sin(\phi) \cdot -\sin(\phi)) = \cos^2(\phi) + \sin^2(\phi) = 1 \]

Therefore,
\[\left( \begin{array}{c} \hat{x} \\ \hat{y} \end{array} \right) = \frac{1}{det} \left(\begin{array}{cc} \cos(\phi) && -\sin(\phi) \\ \sin(\phi) && \cos(\phi) \end{array} \right) \left( \begin{array}{c} \hat{\rho} \\ \hat{\phi} \end{array} \right)\]

and \[\hat{x} = \cos(\phi) \hat{\rho} - \sin(\phi) \hat{\phi}\]
\[\hat{y} = \sin(\phi) \hat{\rho} + \cos(\phi) \hat{\phi}\]

Now we use the Chain Rule is used to calculate the partial derivatives
in terms of \(\rho\) and \(\phi\) based on the conversion identities:
\begin{equation}
\frac {\partial u}{\partial x} = \frac {\partial u}{\partial \rho} \frac {\partial \rho}{\partial x} + \frac {\partial u}{\partial \phi}\frac {\partial \phi}{\partial x}
\label{eq:eq5} \tag{5}
\end{equation}

\begin{equation}
\frac {\partial u}{\partial y} = \frac {\partial u}{\partial \rho} \frac {\partial \rho}{\partial y} + \frac {\partial u}{\partial \phi} \frac {\partial \phi}{\partial y}
\label{eq:eq6} \tag{6}
\end{equation}

Substituting equations x, and y, we get \begin{equation} \label{eq11111}
\begin{split}
\vec{\mathbf{\nabla}} u(x,y) & = \frac{\partial u}{\partial x} \hat{\mathbf{x}} + \frac{\partial u}{\partial y} \hat{\mathbf{y}} \\ & = \left( \frac {\partial u}{\partial \rho}\frac {\partial \rho}{\partial x} + \frac {\partial u}{\partial \phi}\frac {\partial \phi}{\partial x} \right) \left( \cos(\phi) \hat{\rho} - \sin(\phi) \hat{\phi} \right) \\ & + \left(\frac {\partial u}{\partial \rho}\frac {\partial \rho}{\partial y} + \frac {\partial u}{\partial \phi}\frac {\partial \phi}{\partial y} \right)  \left( \sin(\phi) \hat{\rho} + \cos(\phi) \hat{\phi} \right)
\end{split}
\end{equation}

From equations 3a and 3b \begin{equation} \label{eq11112}
\begin{split}
\frac{\partial \rho}{\partial x} & = \frac{\partial }{\partial x} \left( \sqrt{x^2 + y^2} \right) \\ & = \frac{1}{2} \frac{2 x}{\sqrt{x^2 + y^2}} \\ & = \frac{x}{\sqrt{x^2 + y^2}}  \\ & = \frac{x}{\rho} \\ & = \cos(\phi)
\end{split}
\end{equation}

\begin{equation} \label{eq13312}
\begin{split}b =
\frac{\partial \rho}{\partial y} & = \frac{\partial }{\partial y} \left( \sqrt{x^2 + y^2} \right) \\ & = \frac{1}{2} \frac{2 y}{\sqrt{x^2 + y^2}} \\ & = \frac{y}{\sqrt{x^2 + y^2}}  \\ & = \frac{y}{\rho} \\ & = \sin(\phi)
\end{split}
\end{equation}

As for the derivative of \(\tan^{-1}\left( \frac{y}{x} \right)\) Let
\[a = \arctan(b)\] \[\therefore b = \tan(a)\] and
\[db = sec^2(a) \cdot da\] \begin{equation} \label{eq13313}
\begin{split}
\frac{da}{db} & = \frac{1}{\sec^2(a)} \\ & = \frac{1}{1 + \tan^2(a)} \\ & = \frac{1}{1 + b^2} 
\end{split}
\end{equation} or \[da = \frac{1}{1 + b^2} db \]

Since b = \(\frac{y}{x}\) then \begin{equation} \label{eq13314}
\begin{split}
\frac{\partial \phi}{\partial x} & = \frac{1}{1 + \frac{y^2}{x^2}} \cdot \frac{-y}{x^2} \\ & = \frac{x^2}{x^2 + y^2}\cdot \frac{-y}{x^2} \\ & = \frac{-y}{x^2 + y^2} \\ & = \frac{-y}{\rho^2} \\ & = \frac{-y}{\rho}\frac{1}{\rho} \\ & = \frac{-1}{\rho}\sin(\phi)
\end{split}
\end{equation} and similarly, \begin{equation} \label{eq13315}
\begin{split}
\frac{\partial \phi}{\partial y} & = \frac{1}{1 + \frac{y^2}{x^2}} \cdot \frac{1}{x} \\ & = \frac{x^2}{x^2 + y^2}\cdot \frac{1}{x} \\ & = \frac{x}{x^2 + y^2} \\ & = \frac{x}{\rho^2} \\ & = \frac{x}{\rho}\frac{1}{\rho} \\ & = \frac{1}{\rho}\cos(\phi)
\end{split}
\end{equation}

Now we have everything we need to derive the gradient in polar
coordinates.

Substituting equations x, y and z, we get \begin{equation}
\begin{split}
\vec{\mathbf{\nabla}} u(x,y) & = 
\left( \frac {\partial u}{\partial \rho}\frac {\partial \rho}{\partial x} + \frac {\partial u}{\partial \phi}\frac {\partial \phi}{\partial x} \right) \left( \cos(\phi) \hat{\rho} - \sin(\phi) \hat{\phi} \right) \\ & + \left(\frac {\partial u}{\partial \rho}\frac {\partial \rho}{\partial y} + \frac {\partial u}{\partial \phi}\frac {\partial \phi}{\partial y} \right)  \left( \sin(\phi) \hat{\rho} + \cos(\phi) \hat{\phi} \right) 
\\ & =
\left( \frac {\partial u}{\partial \rho} \cos(\phi) + \frac {\partial u}{\partial \phi}\frac {-1}{\rho} \sin(\phi) \right) \left( \cos(\phi) \hat{\rho} - \sin(\phi) \hat{\phi} \right) \\ & + \left(\frac {\partial u}{\partial \rho}\sin(\phi) + \frac {\partial u}{\partial \phi}\frac {1}{\rho} \cos(\phi) \right)  \left( \sin(\phi) \hat{\rho} + \cos(\phi) \hat{\phi} \right)
\\ & = 
\Bigg\{ \frac {\partial u}{\partial \rho} \cos^{2}(\phi)\hat{\rho} - \frac {\partial u}{\partial \rho}\cdot \cos(\phi)\sin(\phi) \hat{\phi}+ \frac{\partial u}{\partial \phi} \left(\frac{-1}{\rho} \sin(\phi) \cos(\phi) \hat{\rho} \right) +  \frac{\partial u}{\partial \phi} \frac{1}{\rho} \sin^{2}(\phi) \hat{\phi} \Bigg\} \\ & + \Bigg\{\frac {\partial u}{\partial \rho}\sin^{2}(\phi)\hat{\rho} + \frac {\partial u}{\partial \rho}\cdot \sin(\phi)\cos(\phi) \hat{\phi} + \frac {\partial u}{\partial \phi}\left( \frac{1}{\rho}\sin(\phi)\cos(\phi)\hat{\rho} \right) + \frac{\partial u}{\partial \phi}\frac{1}{\rho}\cos^{2}(\phi) \hat{\phi} \Bigg\}
\\ & = 
\frac {\partial u}{\partial \rho} \left( \cos^{2}(\phi) + \sin^{2}(\phi) \right) \hat{\rho} + \frac {\partial u}{\partial \phi} \frac{1}{\rho} \left( \cos^{2}(\phi) + \sin^{2}(\phi) \right) \hat{\phi} 
\\ & = 
\frac {\partial u}{\partial \rho} \hat{\rho} + \frac{1}{\rho} \frac {\partial u}{\partial \phi} \hat{\phi} 
\end{split}
\end{equation}

Now that we have the gradient in polar coordinates, we can derive the
LaPlacian using the identity \begin{equation} 
\begin{split}
\nabla^2 u(\rho, \phi) & = \vec{\mathbf{\nabla}} u(\rho, \phi) \cdot \vec{\mathbf{\nabla}} u(\rho, \phi) 
\\ & =
\left( \frac{\partial u}{\partial \rho} \hat{\rho} + \frac{1}{\rho} \frac{\partial u}{\partial \phi} \hat{\phi} \right) \cdot \left( \frac{\partial u}{\partial \rho} \hat{\rho} + \frac{1}{\rho} \frac{\partial u}{\partial \phi} \hat{\phi} \right) 
\\ & =
\frac {\partial u}{\partial \rho} \hat{\rho} \cdot \frac {\partial u}{\partial \rho} \hat{\rho} + 
\frac {\partial u}{\partial \rho} \hat{\rho} \cdot \frac{1}{\rho} \frac {\partial u}{\partial \phi} \hat{\phi} +
\frac{1}{\rho} \frac {\partial u}{\partial \phi} \hat{\phi} \cdot \frac {\partial u}{\partial \rho} \hat{\rho} + 
\frac{1}{\rho} \frac {\partial u}{\partial \phi} \hat{\phi} \cdot \frac{1}{\rho} \frac {\partial u}{\partial \phi} \hat{\phi}
\\ & =
\hat{\rho} \frac {\partial u}{\partial \rho} \cdot \left( \frac {\partial u}{\partial \rho} \hat{\rho} \right) + 
\hat{\rho} \frac {\partial u}{\partial \rho} \cdot \left( \frac{1}{\rho} \frac {\partial u}{\partial \phi} \hat{\phi} \right) +
\hat{\phi} \frac{1}{\rho} \frac {\partial u}{\partial \phi}  \cdot \left (\frac {\partial u}{\partial \rho} \hat{\rho} \right) + 
\hat{\phi} \frac{1}{\rho} \frac {\partial u}{\partial \phi} \cdot \left( \frac{1}{\rho} \frac {\partial u}{\partial \phi} \hat{\phi} \right)
\\ & =
\hat{\rho} \cdot \Bigg\{\frac {\partial u}{\partial \rho} \frac{\partial u}{\partial \rho} \hat{\rho} + \frac {\partial u}{\partial \rho}\frac{\partial {\hat{\rho}}}{\partial \rho} \Bigg\} 
\\ & + 
\hat{\rho} \cdot \Bigg\{\frac {\partial u}{\partial \rho} \left(\frac{1}{\rho} \right) \frac {\partial }{\partial \phi} \hat{\phi} + \frac{1}{\rho}\frac {\partial u}{\partial \rho} \left(\frac {\partial }{\partial \phi} \right)\hat{\phi} + \frac{1}{\rho}\frac {\partial }{\partial \phi}\frac {\partial \hat{\phi}}{\partial \rho}  \Bigg\} 
\\ & + 
\hat{\phi} \frac{1}{\rho} \cdot \Bigg\{\frac {\partial u}{\partial \phi} \left( \frac {\partial u}{\partial \rho} \right) \hat{\rho} +  \frac {\partial u}{\partial \rho} \frac {\partial \hat{\rho}}{\partial \phi} \Bigg\} 
\\ & + 
\hat{\phi} \frac{1}{\rho^2} \cdot \Bigg\{\frac {\partial u}{\partial \phi} \left(\frac {\partial }{\partial \phi} \right) \hat{\phi} + \frac {\partial u}{\partial \phi} \frac {\partial \hat{\phi} }{\partial \phi} \Bigg\}
\\ & =
\hat{\rho} \cdot \Bigg\{\frac {\partial u}{\partial \rho} \left( \frac{\partial u}{\partial \rho} \right) \hat{\rho} + 0 \Bigg\} 
\\ & + 
0
\\ & + 
\hat{\phi} \frac{1}{\rho} \cdot \Bigg\{0 + \frac {\partial u}{\partial \rho} \hat{\phi}\Bigg\} 
\\ & + 
\hat{\phi} \frac{1}{\rho^2} \cdot \Bigg\{\frac {\partial u}{\partial \phi} \frac {\partial }{\partial \phi} \hat{\phi} + 0 \Bigg\}
\\ & =
\frac {\partial^2 u}{\partial \rho^2} \hat{\rho} \cdot \hat{\rho}
\\ & + 
\frac{1}{\rho} \frac {\partial u}{\partial \rho} \hat{\phi} \cdot \hat{\phi}
\\ & + 
\frac{1}{\rho^2} \frac {\partial^2 u}{\partial \phi^2} \hat{\phi} \cdot \hat{\phi}
\\ & =
\frac {\partial^2 u}{\partial \rho^2} + \frac{1}{\rho} \frac {\partial u}{\partial \rho} + 
\frac{1}{\rho^2} \frac {\partial^2 u}{\partial \phi^2} 
\end{split}
\end{equation}

    \hypertarget{separation-of-variables-cylindral-coordinates}{%
\subsection{Separation of Variables Cylindral
Coordinates}\label{separation-of-variables-cylindral-coordinates}}

    We will start with the general time-dependent diffusion equation
\begin{equation}
\nabla^2 u(\rho, \phi, t)  = \frac {\partial u}{\partial t}
\end{equation} in Cyclindrical Coordinates \begin{equation}
\frac {\partial^2 u}{\partial \rho^2} + \frac{1}{\rho} \frac {\partial u}{\partial \rho} + 
\frac{1}{\rho^2} \frac {\partial^2 u}{\partial \phi^2} = \frac {\partial u}{\partial t}
\end{equation}

The radius of the disk will be a.

The boundary conditions are u(a, \(\phi\), t) = constant u(\(\rho\), 0,
t) = u(\(\rho\), 0 + 2 \(\pi\), t)

Leveraging the Separation of Variables (SOR) approach, we will assume
that there is a solution that is the product of 3 functions, each
independent in the 3 variables \(\rho, \phi\) and t. This means
\$u(\rho, \phi, t) = R(\rho) \Phi(\phi) T(t) \$

Substituting the solution into the PDE \begin{equation}
\frac {d^2 R(\rho)}{d \rho^2} \Phi(\phi) T(t) + \frac{1}{\rho} \frac {d R(\rho)}{d \rho} \Phi(\phi) T(t) + 
\frac{1}{\rho^2} \frac {d^2 \Phi}{ \phi^2} R(\rho) T(t) = \frac {d T(t)}{d t} R(\rho) \Phi(\phi) 
\end{equation}

Divide both sides by \(R(\rho) \Phi(\phi) T(t)\) \begin{equation}
\frac {\frac {d^2 R(\rho)}{d \rho^2} \Phi(\phi) T(t) + \frac{1}{\rho} \frac {d R(\rho)}{d \rho} \Phi(\phi) T(t) + 
\frac{1}{\rho^2} \frac {d^2 \Phi}{ \phi^2} R(\rho) T(t)} {R(\rho) \Phi(\phi) T(t)} = \frac{\frac {d T(t)}{d t} R(\rho) \Phi(\phi)} {{R(\rho) \Phi(\phi) T(t)}} 
\end{equation} \begin{equation}
\frac {\frac {d^2 R(\rho)}{d \rho^2} + \frac{1}{\rho} \frac{d R(\rho)}{d \rho}} {R(\rho) }  + 
\frac {\frac{1}{\rho^2} \frac {d^2 \Phi}{ \phi^2} } {\Phi(\phi)} = \frac{\frac {d T(t)}{d t}}{T(t)} 
\end{equation}

We see that the right-hand side of the equation is comprised of t only.
So, it stands that the left-hand side will change at the same rate as
the right-hand side, so we will set it to equal to \$\lambda \$

\begin{equation}
\frac {\frac {d^2 R(\rho)}{d \rho^2} + \frac{1}{\rho} \frac{d R(\rho)}{d \rho}} {R(\rho) }  + 
\frac {\frac{1}{\rho^2} \frac {d^2 \Phi}{ \phi^2} } {\Phi(\phi)} = \frac{\frac {d T(t)}{d t}}{T(t)} = - \lambda
\end{equation}

Solving the ordinary differential equation in time \begin{equation}
\frac{\frac {d T(t)}{d t}}{T(t)} = - \lambda
\end{equation} \begin{equation}
\frac {d T(t)}{d t} + \lambda T(t) = 0
\end{equation}

The solution is going to be \begin{equation}
T(t) = C_{0} e ^ {- \lambda t}
\end{equation}

Before we can create 2 separate ordinary differential equations for
R(\(\rho\)) and \(\Phi(\phi)\) we need to do some work to isolate them.
\begin{equation}
\frac {\frac {d^2 R(\rho)}{d \rho^2} + \frac{1}{\rho} \frac{d R(\rho)}{d \rho}} {R(\rho) }  + 
\frac {\frac{1}{\rho^2} \frac {d^2 \Phi}{d \phi^2} } {\Phi(\phi)} = - \lambda
\end{equation} By rearranging terms \begin{equation}
\frac {\frac {d^2 R(\rho)}{d \rho^2} + \frac{1}{\rho} \frac{d R(\rho)}{d \rho}}{R(\rho)}  + 
\lambda = - \frac{\frac{1}{\rho^2} \frac{d^2 \Phi}{d \phi^2}}{\Phi(\phi)} 
\end{equation} and multiplying both sides by \(\rho^2\) \begin{equation}
\frac {\rho^2 \frac {d^2 R(\rho)}{d \rho^2} + {\rho} \frac{d R(\rho)}{d \rho}} {R(\rho)}  + 
\rho^2 \lambda = - \frac{\frac {d^2 \Phi}{d \phi^2}}{\Phi(\phi)}
\end{equation}

It follows that this implies, in order for the left and right hand sides
to be equal, they must equal a constant that will be assigned to
\(\mu\). \begin{equation}
\rho^2 \frac {1}{R(\rho)} \frac {d^2 R(\rho)}{d \rho^2} + \rho \frac{1}{R(\rho)} \frac{d R(\rho)}{d \rho}   + 
\rho^2 \lambda = - \frac{1}{\Phi(\phi)} \frac{d^2 \Phi}{d \phi^2}  = \mu
\end{equation}

Now we can start by solving the the equation in \(\Phi(\phi)\) first.
\begin{equation}
\frac{1}{\Phi(\phi)} \frac{d^2 \Phi}{d \phi^2}  = -\mu
\end{equation} or \begin{equation}
\frac{d^2 \Phi}{d \phi^2} + \mu \Phi(\phi) = 0
\end{equation}

We have to analyze the following conditions:
\(\mu < 0, \mu = 0, \mu > 0\) given the periodic boundary conditions:
\begin{equation}
\begin{split}
\Phi(- \pi) &= \Phi( \pi) \\
\frac {d\Phi(- \pi)}{d \phi} & = \frac {d \Phi( \pi)}{d \phi}
\end{split}
\end{equation}

For \$\mu = 0 \$ : \begin{equation}
\begin{split}
\frac{d^2 \Phi}{d \phi^2} + \mu \Phi(\phi) = 0 \\
\text{becomes:  } \Phi(\phi) = C_0 + C_1 \phi \\
\text{applying the boundary conditions:  } \\ 
\Phi(\pi) = \Phi(- \pi) \\
C_0 + C_1 \pi = C_0 - C_1 \pi \\
\text{implies } \\
2  C_1 \pi = 0 \\
\text{or } \\
C_1 = 0 \\ 
\text{so there is one solution for $\mu$ = 0: }
\Phi(\phi) = C_0 
\end{split}
\end{equation}-

For \$\mu \textless{} 0 \$ : \begin{equation}
\begin{split}
\frac{d^2 \Phi}{d \phi^2} + \mu \Phi(\phi) &= 0 \\
\Phi(\phi) &= C_2 e^{\sqrt{\mu}\phi} + C_3 e^{\sqrt{\mu}\phi} \\
\text{since $\mu$ < 0, the $\sqrt{\mu}$ is complex:  } \\
\Phi(\phi) &= C_2 (e^{-i \sqrt{\mu}\phi} + e^{i \sqrt{\mu}\phi}) + C_3 (e^{i \sqrt{\mu}\phi} + e^{-i \sqrt{\mu}\phi}) \\
 &= (C_2 + C_3)e^{-i \sqrt{\mu}\phi} +(C_2 + C_3) e^{i \sqrt{\mu}\phi} \\
 &= C_4 e^{-i \sqrt{\mu}\phi} + C_5 e^{i \sqrt{\mu}\phi} \\
 &= C_4 (\cos(\sqrt{\mu}\phi) + i \sin(\sqrt{\mu}\phi)) + C_5 (\cos(\sqrt{\mu}\phi) + i \sin(-\sqrt{\mu}\phi))  \\
 &= (C_4 + C_5) \cos(\sqrt{\mu}\phi) + (C_4 - C_5) i \sin(\sqrt{\mu}\phi) \\
 &= C_6 \cos(\sqrt{\mu}\phi) + C_7 \sin(\sqrt{\mu}\phi) \\
\text{applying the boundary conditions:  } \\
\Phi(\pi) &= \Phi(- \pi) \\
C_6 \cos(\sqrt{\mu}\pi) + C_7 \sin(\sqrt{\mu}\pi) &= C_6 \cos(-\sqrt{\mu}\pi) + C_7 \sin(-\sqrt{\mu}\pi)  \\
\text{applying the identities $\sin(-x)$ = -$\sin(x)$ and $\cos(-x)$ = $\cos(x)$ :  } \\
C_6 \cos(\sqrt{\mu}\pi) + C_7 \sin(\sqrt{\mu}\pi) &= C_6 \cos(\sqrt{\mu}\pi) - C_7 \sin(\sqrt{\mu}\pi)  \\
\text{combining terms:  } \\
2 C_7 \sin(\sqrt{\mu}\pi) &= 0  \\
C_7 \sin(\sqrt{\mu}\pi) &= 0 
\end{split}
\end{equation}

The next step is to determine the non-trivial solutions; that is find
the eigenvalues so that the sine term evaluates to 0. This will happen
when \(\sqrt{\mu}\pi\) is a multiple of \(\pi\) or

\begin{equation}
\begin{split}
\sqrt{\mu}\pi = m \pi  \text{     where m = 1, 2, 3, ... } \\
\text{To simplify, we will set $\sqrt{\mu} = m$ or $\mu = m^2$ } \\
\therefore \\
\Phi(\phi) &= C \sin(m \phi)
\end{split}
\end{equation}

For \$\mu \textgreater{} 0 \$ : \begin{equation}
\begin{split}
\frac{d^2 \Phi}{d \phi^2} + \mu \Phi(\phi) &= 0 \\
\Phi(\phi) &= C_2 e^{\sqrt{\mu}\phi} + C_3 e^{-\sqrt{\mu}\phi} \\
\text{since $\mu$ > 0, the $\sqrt{\mu}$ is real:  } \\
\Phi(\phi) &= C_2 (e^{-\sqrt{\mu}\phi} + e^{\sqrt{\mu}\phi}) + C_3 (e^{\sqrt{\mu}\phi} + e^{-\sqrt{\mu}\phi}) \\
\text{since $\cosh(x)$ = $\frac{e^x + e^{-x}}{2}$} \\
&= 2 C_2 \cosh(\sqrt{\mu}\phi) + 2 C_3 \cosh(\sqrt{\mu}\phi) \\ 
&= C_4 \cosh(\sqrt{\mu}\phi) \\
\text{or }
&= C_4 \cosh(m\phi) \\
\text{applying the boundary condition:  } \\
\Phi(\pi) &= \Phi(-\pi) \\
C_4 \cosh(m\pi) &= C_4 \cosh(-m\pi) \\
\text{since $\cosh(-x) = \cosh(x)$, we do not get a useful answer - only an identity} \\
\text{However, when m $\rightarrow \infty$, $\cosh(x) \rightarrow \infty$ }\\
&= C_4 \cosh(m\phi) \rightarrow \infty \\
\text{and this requires $C_4$ = 0 because the function needs to be bounded.}
\end{split}
\end{equation}

Therefore, after considering the three conditions
\(\mu < 0, \mu = 0, \mu > 0\), and using the superposition:
\begin{equation}
\begin{split}
\Phi(\phi) &= C_0 + C \sin(m \phi) \text{    where m = 1, 2, 3, ...} \\
\text{because each value of m is a solution, we need to sum all of these solutions } \\
\Phi(\phi) &= C_0 + \sum_{m=1}^{\infty}{C_m \sin(m \phi)} 
\end{split}
\end{equation}

We have solved the components for time T(t) and \(\Phi(\phi)\), and the
last step will be to solve the radial equation \(R(rho)\).
\begin{equation}
\begin{split}
\rho^2 \frac {1}{R(\rho)} \frac {d^2 R(\rho)}{d \rho^2} + \rho \frac{1}{R(\rho)} \frac{d R(\rho)}{d \rho}   + 
\rho^2 \lambda &= \mu \\
\rho^2 \frac {1}{R(\rho)} \frac {d^2 R(\rho)}{d \rho^2} + \rho \frac{1}{R(\rho)} \frac{d R(\rho)}{d \rho}   + 
\rho^2 \lambda - \mu &= 0 \\
\text{Multiply through by $R(\rho)$,} \\
\rho^2  \frac {d^2 R(\rho)}{d \rho^2} + \rho \frac{d R(\rho)}{d \rho} + \rho^2 \lambda R(\rho) - \mu R(\rho) &= 0 \\
\rho^2  \frac {d^2 R(\rho)}{d \rho^2} + \rho \frac{d R(\rho)}{d \rho} + (\rho^2 \lambda - \mu)R(\rho) &= 0 \\
\text{since $\mu$ = $m^2$} \\
\rho^2  \frac {d^2 R(\rho)}{d \rho^2} + \rho \frac{d R(\rho)}{d \rho} + (\rho^2 \lambda - m^2)R(\rho) &= 0
\end{split}
\end{equation} The resulting differential equation is a Bessel function.
The derivation can be found here.

The complete solution is therefore: \begin{equation}
\begin{split}
u(\rho, \phi, t) &= R(\rho) \Phi(\phi) T(t) \\
R(\rho) &= J_0(\rho j_{n}) \>\>\>\>\>\>\>\> n = 1, 2, 3, \cdots \\
&= c J_{m}(\rho j_{m,n})  \>\> m, n = 1, 2, 3, \cdots \\
\Phi(\phi) &= \sum_{m=1}^{\infty} C_0 + {C_m \sin(m \phi)} \\
T(t) &= C_{0} e ^ {- \lambda t} = C_{0} e ^ {- j_{m,n}^2 t}  \\
u(\rho, \phi, t) &= \sum_{m=1}^{\infty} \sum_{n=1}^{\infty} c J_{m}(\rho j_{m,n}) \cdot \left(C_0 + {C_m \sin(m \phi)} \right) \cdot  C_{0} e ^ {- \lambda t} \\
&= \sum_{m=1}^{\infty} \sum_{n=1}^{\infty} J_{m}(\rho j_{m,n}) \cdot \left(C_0 + {C_m \sin(m \phi)} \right) \cdot  C_{0} e ^ {- j_{m,n}^2 t} \\
&= \sum_{m=1}^{\infty} \sum_{n=1}^{\infty} C {J_{m}(\rho j_{m,n}) \cdot \left(C_0 + {\sin(m \phi)} \right) \cdot e ^ {- j_{m,n}^2 t} }
\end{split}
\end{equation}

    \hypertarget{derive-the-radial-portion-of-the-heat-equation-in-polar-coordinates.}{%
\subsection{Derive the radial portion of the Heat Equation in Polar
coordinates.}\label{derive-the-radial-portion-of-the-heat-equation-in-polar-coordinates.}}

    We saw the radial components of partial differential equation include a
\(\frac{1}{r}\) term. This means that we have be mindful of how the
center of the circle (\(r = 0\)) affects the overall solution since as
\(r \rightarrow 0\) the solution will approach \(\infty\). Conveniently,
the Frobenius method for solving these types of differential equations
can be employed because it includes a displacement factor around the
singularity.

\begin{equation}
\begin{split} 
\rho^2 \frac {d^2 R(\rho)}{d \rho^2} + \rho \frac{d R(\rho)}{d\rho} + (\rho^2 \lambda - m^2)R(\rho) &= 0 \\
\text{Substitue $\rho = r \sqrt \lambda$ such that $R(\rho) \rightarrow \overline{R}(r \sqrt \lambda)$}  \\
r^2 \frac {d^2 \overline{R}(r)}{dr^2} + r\frac{d \overline{R}(r)}{dr} + (r^2 - m^2)\overline{R}(r) &= 0 
\end{split}
\end{equation} and the boundary conditions transform to \begin{equation}
\begin{split} 
| \overline{R}(0) | \lt \infty \\
\overline{R}(a \sqrt{\lambda}) = T
\end{split}
\end{equation}

Start with the Frobenius equation: \begin{equation}
\begin{split}
\overline{R}(r) & = \sum_{n=0}^{\infty} a_n r^{\alpha + n}
\end{split}
\end{equation}

Subtituting this into the ODE \begin{equation}
\begin{split}
\overline{R}(r) &= r^2 \sum_{n=0}^{\infty} a_{n} (\alpha + n) (\alpha + n - 1) r^{\alpha + n - 2} + r \sum_{n=0}^{\infty} a_{n} (\alpha + n) r^{\alpha + n - 1} + (r^2 - m^2)\sum_{n=0}^{\infty} a_n r^{\alpha + n} \\
&= r^2 \sum_{n=0}^{\infty} a_n (\alpha + n) (\alpha + n - 1) r^{\alpha + n - 2} + r \sum_{n=0}^{\infty} a_{n} (\alpha + n) r^{\alpha + n - 1} + (r^2 - m^2)\sum_{n=0}^{\infty} a_n r^{\alpha + n} \\
&= \sum_{n=0}^{\infty} a_{n} (\alpha + n) (\alpha + n - 1) r^{\alpha + n} + \sum_{n=0}^{\infty} a_{n} (\alpha + n) r^{\alpha + n} + (r^2 - m^2)\sum_{n=0}^{\infty} a_n r^{\alpha + n} \\
&= \sum_{n=0}^{\infty} a_{n} (\alpha + n) (\alpha + n - 1) r^{\alpha + n} + \sum_{n=0}^{\infty} a_{n} (\alpha + n) r^{\alpha + n} + (r^2 - m^2)\sum_{n=0}^{\infty} a_n r^{\alpha + n} \\
&= \sum_{n=0}^{\infty} a_{n} (\alpha + n) (\alpha + n - 1) r^{\alpha + n} + \sum_{n=0}^{\infty} a_{n} (\alpha + n) r^{\alpha + n} + \sum_{n=0}^{\infty}{a_n r^{\alpha + n + 2}} - m^2 \sum_{n=0}^{\infty}{a_n r^{\alpha + n}}\\
&= \sum_{n=0}^{\infty} a_{n} (\alpha + n) (\alpha + n - 1) r^{\alpha + n} + \sum_{n=0}^{\infty} a_{n} (\alpha + n) r^{\alpha + n} + \sum_{n=2}^{\infty}{a_{n-2} r^{\alpha + n}} - m^2 \sum_{n=0}^{\infty}{a_n r^{\alpha + n}}\\
&= \sum_{n=0}^{\infty} \left[ a_{n} (\alpha + n) (\alpha + n - 1) r^{\alpha + n} +
a_{n} (\alpha + n) r^{\alpha + n} - m^2 a_n r^{\alpha + n} \right] + \sum_{n=2}^{\infty}{a_{n-2} r^{\alpha + n}} \\ 
&= \sum_{n=0}^{\infty} \left[ a_{n} (\alpha + n) (\alpha + n - 1)  +
a_{n} (\alpha + n) - m^2 a_n  \right] r^{\alpha + n} + \sum_{n=2}^{\infty}{a_{n-2} r^{\alpha + n}} = 0
\end{split}
\end{equation}

The next step will be to find the indicial equation from which, we will
find a pattern (hopefully), to transform the infinite series into a more
easily discernable equation. As stated in equation 1, the differential
equation is equal to 0. In turn, this means that the sum of the
coefficients of the sum must be zero.

To find the inidicial equation we set n=0. The second term in (3) will
not participate, until n=2.

\begin{equation}
\begin{split}
a_{0} (\alpha) (\alpha - 1)  + a_{0} \alpha - m^2 a_0 &= 0 \\
a_{0} \left((\alpha) (\alpha - 1)  + \alpha - m^2 \right) &= 0 \\
a_{0} \left(\alpha^{2} - m^2 \right) &= 0 \\
\end{split}
\end{equation}

The non-trivial solution will be when

\begin{equation}
\begin{split}
\alpha^{2} &= m^{2} \\
\text{or} \\
\alpha &= \pm m
\end{split}
\end{equation}

If we take the positive m such that \$\alpha = +m \$ and substitute it
back into (3) then \begin{equation}
\begin{split}
\sum_{n=0}^{\infty} \left[ a_{n} (m + n) (m + n - 1)  +
a_{n} (m + n) - m^2 a_n  \right] r^{m + n} + \sum_{n=2}^{\infty}{a_{n-2} r^{m + n}} &= \\
\sum_{n=0}^{\infty} \left[ a_{n} n(n + 2m) \right] r^{m + n} + \sum_{n=2}^{\infty}{a_{n-2} r^{m + n}} &= 0 
\end{split}
\end{equation}

Setting n=1 leads to the following \begin{equation}
\begin{split}
\left[ a_{1} (2m + 1) \right] r^{m + 1} + \sum_{n=2}^{\infty}{a_{n-2} r^{m + n}} &= 0 \\
\text{the right-hand term does not participate, so the left-hand term above must be zero} \\
\left[ a_{1} (2m + 1) \right] r^{m + 1}  &= 0
\end{split}
\end{equation} This means that either \(a_{1}\) is zero or 2m + 1 = 0.
We know that m is an integer and \(m \geq 0\) when we solved the angular
portion of the LaPlacian. So, \(a_{1} = 0\) and, this means that all odd
terms will be zero in this recursion.

If that were not the case we would need to consider m =
\(-\frac{1}{2}\). This solution is covered at qqqqqq.

Now we can look at the \(n \geq 2\) terms. \begin{equation}
\begin{split}
\sum_{n=2}^{\infty} \left[ a_{n} (\alpha + n) (\alpha + n - 1)  +
a_{n} (\alpha + n) - m^2 a_n  \right] r^{\alpha + n} + \sum_{n=2}^{\infty}{a_{n-2} r^{\alpha + n}} &= 0 \\
\sum_{n=2}^{\infty} \left[ a_{n} (\alpha + n) (\alpha + n - 1)  +
a_{n} (\alpha + n) - m^2 a_n  + {a_{n-2}}\right]r^{\alpha + n} &= 0 \\
\left[ a_{n} (\alpha + n) (\alpha + n - 1)  + a_{n} (\alpha + n) - m^2 a_n  + {a_{n-2}}\right] &= 0 \\
a_{n} \left[(\alpha + n) (\alpha + n - 1) + (\alpha + n) - m^2 \right]  + {a_{n-2}} &= 0 \\
a_{n} \left[(\alpha + n) \left((\alpha + n - 1) + 1 \right) - m^2 \right]  + {a_{n-2}} &= 0 \\
a_{n} \left[(\alpha + n) \left((\alpha + n \right) - m^2 \right]  + {a_{n-2}} &= 0 \\
a_{n} \left[(\alpha + n)^2 - m^2 \right]  + {a_{n-2}} &= 0 \\
\text{For now, we will work with $\alpha = m$} \\
a_{n} \left[(m + n)^2 - m^2 \right]  + {a_{n-2}} &= 0 \\
a_{n} \left[m^2 + 2 m n + n^2 - m^2 \right]  + {a_{n-2}} &= 0 \\
a_{n} \left[2 m n + n^2 \right]  + {a_{n-2}} &= 0 \\
a_{n} \left[n (2 m + n) \right]  + {a_{n-2}} &= 0 \\
a_{n} \left[n (2 m + n) \right]  &= - {a_{n-2}}  \\
a_{n} &= - \frac{1}{n (2 m + n)} a_{n-2}
\end{split}
\end{equation}

Recall that all odd terms are equal to zero. A new index scheme can be
used so that all terms can be represented with a single index. Let the
odd terms be represented by n = 2 b + 1 and the even terms represented
by n = 2 b where b = 0, 1, 2, 3, \ldots{}

Then, the odd terms are defined by \begin{equation}
\begin{split}
a_{2b + 1} &= - \frac{a_{2 b + 1 - 2}}{(2b + 1)(2 m + ( 2 b + 1))} \\
&= - \frac{a_{2 b - 1}}{(2b + 1)(2 m + 2 b + 1)} \\
&= - \frac{a_{2 b - 1}}{(2b + 1)(2 (m + b) + 1)}
\end{split}
\end{equation}

and the even terms \begin{equation}
\begin{split}
a_{2b} &= - \frac{a_{2 b - 2}}{(2b)(2 m + 2 b)} \\
&= - \frac{a_{2 b - 2}}{(2b)(2 m + 2 b)} \\
&= - \frac{a_{2 b - 2}}{(2b)(2) (m + b)} \\
&= - \frac{a_{2 b - 2}}{(2^2)(b)(m + b)}
\end{split}
\end{equation}

Let's look at the third term (b=3) in this recursive series \(a_{6}\) to
start teasing out a pattern.\\
\begin{equation}
\begin{split}
a_{2b} &= - \frac{1 }{(2^2)(b)(m + b)} {a_{2 b - 2}} \\
a_{6} &= - \frac{1 }{(2^2)(3)(m + 3)} {a_{2 \cdot 3 - 2}} \\
&= - \frac{1 }{(2^2)(3)(m + 3)} {a_{4}} \\
&= - \frac{- 1 }{(2^2)(3)(m + 3)} \cdot \frac{- 1} {(2^2)(2)(m + 2)} a_{2} \\
&= - \frac{- 1 }{(2^2)(3)(m + 3)} \cdot \frac{- 1} {(2^2)(2)(m + 2)} \cdot \frac{- 1 }{(2^2)(1)(m + 1)} a_{0} \\
&= \frac{- 1 \cdot - 1 \cdot - 1  }{(2^2)(2^2)(2^2) \cdot (3)(2)(1) \cdot (m + 3)(m + 2)(m + 1)} a_{0} \\
&= \frac{(-1)^3}{(2^{2 \cdot 3}) \cdot 3! \cdot (m + 3)(m + 2)(m + 1)} a_{0}
\end{split}
\end{equation}

Let's look at the (m + 3)(m + 2)(m + 1) product. This looks like a few
terms of a (m + 3)! but there are a few terms missing. For b=3 (m + b)!
is \begin{equation}
\begin{split}
(m + 3)! &= (m + 3) \cdot (m + (3 - 1)) \cdot (m + (3 - 2)) \cdot (m + (3 - 3)) \cdot (m - 1) \cdot (m - 2) \cdot (m - 3) \cdots (m - (m - 1))\\
&= (m + 3) \cdot (m + 2) \cdot (m + 1) \cdot \underbrace{m \cdot (m - 1) \cdot (m - 2) \cdot (m - 3) \cdots (1)}_{m!} \\
&= (m + 3) \cdot (m + 2) \cdot (m + 1) \cdot m! \\
\therefore (m + 3) \cdot (m + 2) \cdot (m + 1) &= \frac{(m + 3)!}{m!}
\end{split}
\end{equation}

Substiuting this result into the relationship that defines \(a_{2b}\),
\begin{equation}
\begin{split}
a_{2b} &= \frac{ (-1)^3}{(2^{2 \cdot 3}) \cdot 3! \cdot (m + 3)(m + 2)(m + 1)} a_{0} \\
&= \frac{ (-1)^3}{(2^{2 \cdot 3}) \cdot 3! \cdot \frac{(m + 3)!}{m!}} a_{0} \\
&= \frac{ (-1)^3 m!}{(2^{2 \cdot 3}) \cdot 3! \cdot (m + 3)!} a_{0}
\end{split}
\end{equation}

Generalizing this result \begin{equation}
\begin{split}
a_{2b} &= \frac{(-1)^b m!}{(2^{2 \cdot b}) \cdot b! \cdot (m + b)!} a_{0}
\end{split}
\end{equation}

Finally, we have \begin{equation}
\begin{split} 
\overline{R}(r) &= \sum_{n=0}^{\infty} a_n r^{m + n} \\
&= \sum_{n=1,3,5 \cdots}^{\infty} a_{n} r^{m + n} + \sum_{n=0,2,4\cdots}^{\infty} a_{n} r^{m + n} \\
&= \sum_{n=1,3,5 \cdots}^{\infty} a_{2b+1} r^{m + (2b + 1)} + \sum_{n=0,2,4\cdots}^{\infty} a_{2b} r^{m + (2b)} \\
&= \sum_{b=0}^{\infty} a_{2b+1} r^{m + (2b + 1)} + \sum_{b=0}^{\infty} a_{2b} r^{m + (2b)} \\
\text{We showed that all odd terms are zero:}\\
&= \underbrace {\sum_{b=0}^{\infty} a_{2b+1} r^{m + (2b + 1)}}_{0} + \sum_{b=0}^{\infty} a_{2b} r^{m + (2b)} \\
&= \sum_{b=0}^{\infty} a_{2b} r^{m + (2b)} \\
&= \sum_{b=0}^{\infty} \frac{(-1)^b m!}{(2^{2b}) \cdot b! \cdot (m + b)!} a_{0} r^{m + 2b} \\
&= \sum_{b=0}^{\infty} \frac{(-1)^b m!}{(2^{2b}) \cdot b! \cdot (m + b)!} a_{0} r^{m + 2b} \\
&= r^m\sum_{b=0}^{\infty} \frac{(-1)^b m!}{(2^{2b}) \cdot b! \cdot (m + b)!} a_{0} r^{2b}
\end{split}
\end{equation}

Recall Bessel's power series function \begin{equation}
J_{m}(x) = \sum_{k=0}^{\infty} \frac{-1^k}{2^{2k+m} k! (k + m)!} x^{2k + m}
\end{equation}

The previous recursion equation can then be rewritten as
\begin{equation}
\begin{split} 
\overline{R}(r) &= \sum_{b=0}^{\infty} \frac{(-1)^b m!}{(2^{2b}) \cdot b! \cdot (m + b)!} a_{0} r^{2b + m}  \\
&= \sum_{b=0}^{\infty} \frac{(-1)^b m!}{(2^{2b}) \cdot b! \cdot (m + b)!} a_{0} \cdot \frac{2^m}{2^m} r^{2b + m}  \\
&= a_{0}  m! 2^m \underbrace{\sum_{b=0}^{\infty} \frac{(-1)^b}{(2^{2b + m}) \cdot b! \cdot (m + b)!} r^{2b + m}}_{J_m(r)}   \\
&= c J_{m}(r)
\end{split} 
\end{equation}

Substituting \(\rho\) for r \begin{equation}
\begin{split} 
{R}(\rho) &= c J_{m}(\rho \sqrt \lambda) = c J_{m}(\rho j_{m,n}) \\
\lambda_{m,n} &= j_{m,n}^2 \\
m, n &= 0, 1, 2 \cdots
\end{split} 
\end{equation}

    \begin{tcolorbox}[breakable, size=fbox, boxrule=1pt, pad at break*=1mm,colback=cellbackground, colframe=cellborder]
\prompt{In}{incolor}{4}{\boxspacing}
\begin{Verbatim}[commandchars=\\\{\}]
\PY{c+c1}{\PYZsh{} Plot first 5 Bessel functions}
\PY{k+kn}{import} \PY{n+nn}{numpy} \PY{k}{as} \PY{n+nn}{np}
\PY{k+kn}{from} \PY{n+nn}{matplotlib} \PY{k+kn}{import} \PY{n}{pyplot} \PY{k}{as} \PY{n}{plt}
\PY{k+kn}{import} \PY{n+nn}{scipy}\PY{n+nn}{.}\PY{n+nn}{special} \PY{k}{as} \PY{n+nn}{spl} 

\PY{c+c1}{\PYZsh{} Generating time data using arange function from numpy}
\PY{n}{x} \PY{o}{=} \PY{n}{np}\PY{o}{.}\PY{n}{linspace}\PY{p}{(}\PY{l+m+mi}{0}\PY{p}{,} \PY{l+m+mi}{30}\PY{p}{,} \PY{l+m+mi}{1000}\PY{p}{)}

\PY{k}{for} \PY{n}{i} \PY{o+ow}{in} \PY{n+nb}{range}\PY{p}{(}\PY{l+m+mi}{0}\PY{p}{,} \PY{l+m+mi}{5}\PY{p}{)}\PY{p}{:}
    \PY{n}{J} \PY{o}{=} \PY{n}{spl}\PY{o}{.}\PY{n}{jv}\PY{p}{(}\PY{n}{i}\PY{p}{,} \PY{n}{x}\PY{p}{)}
    \PY{n}{plt}\PY{o}{.}\PY{n}{plot}\PY{p}{(}\PY{n}{x}\PY{p}{,} \PY{n}{J}\PY{p}{,} \PY{n}{label} \PY{o}{=} \PY{l+s+sa}{r}\PY{l+s+s1}{\PYZsq{}}\PY{l+s+s1}{\PYZdl{}J\PYZus{}}\PY{l+s+s1}{\PYZsq{}} \PY{o}{+} \PY{n+nb}{str}\PY{p}{(}\PY{n}{i}\PY{p}{)} \PY{o}{+} \PY{l+s+s1}{\PYZsq{}}\PY{l+s+s1}{(x)\PYZdl{}}\PY{l+s+s1}{\PYZsq{}}\PY{p}{)}

\PY{n}{plt}\PY{o}{.}\PY{n}{legend}\PY{p}{(}\PY{p}{)}

\PY{c+c1}{\PYZsh{} Settng title for the plot in blue color}
\PY{n}{plt}\PY{o}{.}\PY{n}{title}\PY{p}{(}\PY{l+s+s1}{\PYZsq{}}\PY{l+s+s1}{Modified Bessel \PYZhy{} First Kind}\PY{l+s+s1}{\PYZsq{}}\PY{p}{,} \PY{n}{color}\PY{o}{=}\PY{l+s+s1}{\PYZsq{}}\PY{l+s+s1}{r}\PY{l+s+s1}{\PYZsq{}}\PY{p}{)}

\PY{c+c1}{\PYZsh{} Setting x axis label for the plot}
\PY{n}{plt}\PY{o}{.}\PY{n}{xlabel}\PY{p}{(}\PY{l+s+s1}{\PYZsq{}}\PY{l+s+s1}{x}\PY{l+s+s1}{\PYZsq{}}\PY{o}{+} \PY{l+s+sa}{r}\PY{l+s+s1}{\PYZsq{}}\PY{l+s+s1}{\PYZdl{}}\PY{l+s+s1}{\PYZbs{}}\PY{l+s+s1}{rightarrow\PYZdl{}}\PY{l+s+s1}{\PYZsq{}}\PY{p}{)}

\PY{c+c1}{\PYZsh{} Setting x axis label for the plot}
\PY{n}{plt}\PY{o}{.}\PY{n}{ylabel}\PY{p}{(}\PY{l+s+s1}{\PYZsq{}}\PY{l+s+s1}{\PYZdl{}J\PYZus{}v(x)\PYZdl{}}\PY{l+s+s1}{\PYZsq{}}\PY{o}{+} \PY{l+s+sa}{r}\PY{l+s+s1}{\PYZsq{}}\PY{l+s+s1}{\PYZdl{}}\PY{l+s+s1}{\PYZbs{}}\PY{l+s+s1}{rightarrow\PYZdl{}}\PY{l+s+s1}{\PYZsq{}}\PY{p}{)}

\PY{c+c1}{\PYZsh{} Showing grid}
\PY{n}{plt}\PY{o}{.}\PY{n}{grid}\PY{p}{(}\PY{p}{)}

\PY{c+c1}{\PYZsh{} Highlighting axis at x=0 and y=0}
\PY{n}{plt}\PY{o}{.}\PY{n}{axhline}\PY{p}{(}\PY{n}{y}\PY{o}{=}\PY{l+m+mi}{0}\PY{p}{,} \PY{n}{color}\PY{o}{=}\PY{l+s+s1}{\PYZsq{}}\PY{l+s+s1}{b}\PY{l+s+s1}{\PYZsq{}}\PY{p}{)}
\PY{n}{plt}\PY{o}{.}\PY{n}{axvline}\PY{p}{(}\PY{n}{x}\PY{o}{=}\PY{l+m+mi}{0}\PY{p}{,} \PY{n}{color}\PY{o}{=}\PY{l+s+s1}{\PYZsq{}}\PY{l+s+s1}{b}\PY{l+s+s1}{\PYZsq{}}\PY{p}{)}

\PY{c+c1}{\PYZsh{} Finally displaying the plot}
\PY{n}{plt}\PY{o}{.}\PY{n}{show}\PY{p}{(}\PY{p}{)}
\end{Verbatim}
\end{tcolorbox}

    
    \begin{Verbatim}[commandchars=\\\{\}]
<IPython.core.display.Javascript object>
    \end{Verbatim}

    
    
    \begin{Verbatim}[commandchars=\\\{\}]
<IPython.core.display.HTML object>
    \end{Verbatim}

    
    Zeros of the first 5 Bessel functions can be found in the table below.
The table is read across for each \(J_m\) as the eigenvalue
\(j_{m,0}, j_{m,1}, j_{m,2} \cdots\) where m = {[}0, 4{]}

    \begin{tcolorbox}[breakable, size=fbox, boxrule=1pt, pad at break*=1mm,colback=cellbackground, colframe=cellborder]
\prompt{In}{incolor}{3}{\boxspacing}
\begin{Verbatim}[commandchars=\\\{\}]
\PY{c+c1}{\PYZsh{}Find the first 5 values where each Bessel function is zero (i.e. calculate the eigenvalues)}
\PY{k+kn}{import} \PY{n+nn}{scipy}\PY{n+nn}{.}\PY{n+nn}{special} \PY{k}{as} \PY{n+nn}{spl} 
\PY{k+kn}{import} \PY{n+nn}{pandas} \PY{k}{as} \PY{n+nn}{pd} 
\PY{k+kn}{import} \PY{n+nn}{numpy} \PY{k}{as} \PY{n+nn}{np}

\PY{n}{row\PYZus{}labels} \PY{o}{=} \PY{p}{[}\PY{l+s+s1}{\PYZsq{}}\PY{l+s+s1}{J0}\PY{l+s+s1}{\PYZsq{}}\PY{p}{,} \PY{l+s+s1}{\PYZsq{}}\PY{l+s+s1}{J1}\PY{l+s+s1}{\PYZsq{}}\PY{p}{,} \PY{l+s+s1}{\PYZsq{}}\PY{l+s+s1}{J2}\PY{l+s+s1}{\PYZsq{}}\PY{p}{,} \PY{l+s+s1}{\PYZsq{}}\PY{l+s+s1}{J3}\PY{l+s+s1}{\PYZsq{}}\PY{p}{,} \PY{l+s+s1}{\PYZsq{}}\PY{l+s+s1}{J4}\PY{l+s+s1}{\PYZsq{}}\PY{p}{]}
\PY{n}{data} \PY{o}{=} \PY{p}{\PYZob{}}\PY{l+s+s1}{\PYZsq{}}\PY{l+s+s1}{first}\PY{l+s+s1}{\PYZsq{}}\PY{p}{:}\PY{p}{[}\PY{p}{]}\PY{p}{,} \PY{l+s+s1}{\PYZsq{}}\PY{l+s+s1}{second}\PY{l+s+s1}{\PYZsq{}}\PY{p}{:} \PY{p}{[}\PY{p}{]}\PY{p}{,} \PY{l+s+s1}{\PYZsq{}}\PY{l+s+s1}{third}\PY{l+s+s1}{\PYZsq{}}\PY{p}{:} \PY{p}{[}\PY{p}{]}\PY{p}{,} \PY{l+s+s1}{\PYZsq{}}\PY{l+s+s1}{fourth}\PY{l+s+s1}{\PYZsq{}}\PY{p}{:}\PY{p}{[}\PY{p}{]}\PY{p}{,} \PY{l+s+s1}{\PYZsq{}}\PY{l+s+s1}{fifth}\PY{l+s+s1}{\PYZsq{}}\PY{p}{:} \PY{p}{[}\PY{p}{]}\PY{p}{\PYZcb{}}
\PY{n}{i}\PY{o}{=}\PY{l+m+mi}{0}
\PY{k}{for} \PY{n}{key}\PY{p}{,} \PY{n}{value} \PY{o+ow}{in} \PY{n}{data}\PY{o}{.}\PY{n}{items}\PY{p}{(}\PY{p}{)}\PY{p}{:}
    \PY{n}{data}\PY{p}{[}\PY{n}{key}\PY{p}{]} \PY{o}{=} \PY{n}{np}\PY{o}{.}\PY{n}{array}\PY{p}{(}\PY{n}{spl}\PY{o}{.}\PY{n}{jn\PYZus{}zeros}\PY{p}{(}\PY{n}{i}\PY{p}{,} \PY{l+m+mi}{5}\PY{p}{)}\PY{p}{)}
    \PY{n}{i}\PY{o}{=}\PY{n}{i}\PY{o}{+}\PY{l+m+mi}{1}
    
\PY{k}{with} \PY{n}{pd}\PY{o}{.}\PY{n}{option\PYZus{}context}\PY{p}{(}\PY{l+s+s1}{\PYZsq{}}\PY{l+s+s1}{display.float\PYZus{}format}\PY{l+s+s1}{\PYZsq{}}\PY{p}{,} \PY{l+s+s1}{\PYZsq{}}\PY{l+s+si}{\PYZob{}:3.3f\PYZcb{}}\PY{l+s+s1}{\PYZsq{}}\PY{o}{.}\PY{n}{format}\PY{p}{)}\PY{p}{:}    
    \PY{n}{df} \PY{o}{=} \PY{n}{pd}\PY{o}{.}\PY{n}{DataFrame}\PY{p}{(}\PY{n}{data}\PY{p}{,} \PY{n}{index}\PY{o}{=}\PY{n}{row\PYZus{}labels}\PY{p}{)}  
    \PY{n+nb}{print}\PY{p}{(}\PY{n}{df}\PY{p}{)}
\end{Verbatim}
\end{tcolorbox}

    \begin{Verbatim}[commandchars=\\\{\}]
    first  second  third  fourth  fifth
J0  2.405   3.832  5.136   6.380  7.588
J1  5.520   7.016  8.417   9.761 11.065
J2  8.654  10.173 11.620  13.015 14.373
J3 11.792  13.324 14.796  16.223 17.616
J4 14.931  16.471 17.960  19.409 20.827
    \end{Verbatim}

    \hypertarget{how-to-integrate-int-cosaxcosbx-dx}{%
\subsection{\texorpdfstring{How to integrate
\(\int cos(ax)cos(bx) dx\)}{How to integrate \textbackslash int cos(ax)cos(bx) dx}}\label{how-to-integrate-int-cosaxcosbx-dx}}

    One approach to solving this type of integral is to start with the
Product Rule
\[\frac {d}{dx} (u(x) v(x)) = v(x) \frac {du(x)}{dx} + u(x) \frac {dv(x)}{dx}\].

Let's get started:

Pull one of the constants out of one of the cosine terms. For this
example, we will use the variable a

\begin{equation}
I = \int cos(ax)cos(bx) dx
\end{equation}

Notice that \[\frac {d (sin(ax))}{dx} = a \cdot cos(ax)\] or
\begin{equation}
cos(ax) = \frac {1}{a} \frac {d(sin(ax))}{dx}
\label{eq:eq2} \tag{2}
\end{equation}

Substitute the equality in \#2 into the original integral to get
\[I = \frac{1}{a}\int cos(bx) (\frac {d(sin(ax))}{dx}){dx}\]

Set \[u(x) = cos(bx) \to\ \frac{d u(x)}{dx} = -b sin(bx)\] and
\[\frac {d (v(x))}{dx} = \frac {d (sin(ax))}{dx} \to\ v(x) = sin(ax)\]

Make the necessary subsitutions into the Product Rule and integrate both
sides: So \[\frac{1}{a}\int cos(bx) (\frac {d(sin(ax))}{dx}){dx}\] with
the following
\[\frac {1}{a} \int \frac {d}{dx} (u(x) v(x)) = \frac {1}{a} \int v(x) \frac {du(x)}{dx} + \frac {1}{a} \int u(x) \frac {dv(x)}{dx}\]
becomes \begin{equation}
\frac {1}{a} cos(bx) sin(ax) = \frac {1}{a} \int -b \cdot sin(bx) sin(ax) dx + \frac {1}{a} \int cos(bx) \frac {d(sin(ax))}{dx}
\label{eq:eq3} \tag{3}
\end{equation}

Rearrange the terms so that the original integral is isolated:
\[\frac {1}{a} \int cos(bx) \frac {d(sin(ax))}{dx} = \frac {1}{a} cos(bx) sin(ax) + \frac {1}{a} \int b \cdot sin(bx) sin(ax) dx \]
or \begin{equation}
I = \frac {1}{a} cos(bx) sin(ax) + \frac {1}{a} \int b \cdot sin(bx) sin(ax) dx 
\label{eq:eq4} \tag{4}
\end{equation}

Repeat the previous steps for the integral on the right-hand side using
the Product Rule again.\\
\[\frac {1}{a} \int b \cdot sin(bx) sin(ax) dx \] Notice that
\[\frac {d (sin(ax))}{dx} = a \cdot sin(ax)\] or
\[a \cdot sin(ax) = \frac {1}{a} \frac {d(-cos(ax))}{dx}\] Or
\[\frac {1}{a} \int b \cdot sin(bx) sin(ax) dx = \frac {1}{a} \int b \cdot sin(bx) \frac {1}{a} \frac {d(-cos(ax))}{dx} dx \]
Create the following identities
\[u(x) = sin(bx) \to\ \frac{d u(x)}{dx} = b \cdot cos(bx)\] and
\[\frac {d (v(x))}{dx} = \frac {-d (cos(ax))}{dx} \to\ v(x) = cos(ax)\]

Using the Product Rule again \$\frac {1}{a} \int b \cdot sin(bx) sin(ax)
dx \$ becomes \begin{equation}
-\frac{1}{a} [ sin(bx) cos(ax) - b \int cos(ax) cos(bx) dx]
\label{eq:eq5} \tag{5}
\end{equation} Since \$I = \int cos(ax) cos(bx) dx \$ then equation 5
becomes \begin{equation}
-\frac{1}{a} [ sin(bx) cos(ax) - b \cdot I ]
\label{eq:eq6} \tag{6}
\end{equation}

From equation 4
\[I = \frac {1}{a} cos(bx) sin(ax) + \frac {1}{a} \int b \cdot sin(bx) sin(ax) dx \]
and substituting equation 6 for the right-hand term

\begin{eqnarray*}
I & = & \frac {1}{a}sin(ax) cos(bx) + \frac{b}{a} \left(- \frac{1}{a} sin(bx) cos(ax) + \frac {b}{a} I + C_2 \right) \\
& = & \frac {1}{a}sin(ax) cos(bx) - \frac{b}{a^2} sin(bx) cos(ax) + \frac {b^2}{a^2} I + C_2 
\end{eqnarray*}

Rearrange terms to isolate I
\[\frac{a^2-b^2}{a^2} I = \frac {1}{a}sin(ax) cos(bx) - \frac{b}{a^2} sin(bx) cos(ax) + C_2\]
or
\[I = \frac{a^2}{a^2-b^2} \left( \frac {1}{a}sin(ax) cos(bx) - \frac{b}{a^2} sin(bx) cos(ax) + C_2 \right)\]
and simplifying
\[I = \frac{a^2}{a^2-b^2} \left( \frac {a \cdot sin(ax) cos(bx) - b \cdot sin(bx) cos(ax)}{a^2} + C_2 \right)\]

\begin{equation}
I = \frac{1}{a^2-b^2} \left(a \cdot sin(ax) cos(bx) - b \cdot sin(bx) cos(ax) + C_2 \right)
\label{eq:eq7} \tag{7}
\end{equation}

NOTE that \[{a^2} \neq {b^2}\]

Use the following trigonmetric identities to simplify the right-hand
side further \begin{equation}
cos(a)sin(b) = \frac{1}{2}sin(a+b) - \frac{1}{2}sin(a-b)
\label{eq:eq8} \tag{8}
\end{equation} \begin{equation}
sin(a)cos(b) = \frac{1}{2}sin(a+b) + \frac{1}{2}sin(a-b)
\label{eq:eq9} \tag{9}
\end{equation}

Now \(a \cdot sin(ax) cos(bx) - b \cdot sin(bx) cos(ax)\) can be written
as follows:
\[\frac{1}{2} a \cdot sin(a+b)x + \frac{1}{2}a \cdot sin(a-b)x +\frac{1}{2}b \cdot sin(a+b)x - \frac{1}{2}b \cdot sin(a-b)x\]

rearranging the terms \begin{equation}
\frac{1}{2} \left[(a+b) \cdot sin(a-b)x + (a-b) \cdot sin(a+b)x \right)
\label{eq:eq10} \tag{10}
\end{equation} equation 7 becomes

\begin{equation}
I = \frac{1}{a^2 - b^2} \left(\frac{1}{2} \left[(a+b) \cdot sin(a-b)x + (a-b) \cdot sin(a+b)x \right] \right)
\label{eq:eq11} \tag{11}
\end{equation}

Since \({a^2} - {b^2} = (a-b)(a+b)\) equation 11 becomes
\begin{equation}
I = \frac{sin(a-b)x}{2(a-b)} + \frac{sin(a+b)x}{2(a+b)}
\label{eq:eq12} \tag{12}
\end{equation}

    \hypertarget{how-to-integrate-int-sinaxsinbx-dx}{%
\subsection{\texorpdfstring{How to integrate
\(\int sin(ax)sin(bx) dx\)}{How to integrate \textbackslash int sin(ax)sin(bx) dx}}\label{how-to-integrate-int-sinaxsinbx-dx}}

    One approach to solving this type of integral is to start with the
Product Rule
\[\frac {d}{dx} (u(x) \cdot v(x)) = v(x) \frac {du(x)}{dx} + u(x) \frac {dv(x)}{dx} .\]

\begin{enumerate}
\def\labelenumi{\arabic{enumi}.}
\tightlist
\item
  Start by defining a variable \(I = \int sin(ax)sin(bx) dx\) as these
  types of integrals repeat during the derivation process.
\item
  Note that \[\frac {d (-cos(ax))}{dx} = a \cdot sin(ax)\] or
  \[sin(ax) = \frac {1}{a} \frac {d(cos(ax))}{dx} .\]
\end{enumerate}

This is required to get the integral in the same form as the Product
Rule. 3. Substitute the equality from \#2 into the original integral to
get \[I = \frac{1}{a}\int sin(bx) (\frac {d(-cos(ax))}{dx}){dx}\] 4. Set
the following identities to align with the Product Rule terms:
\[u(x) = sin(bx) \to\ \frac{d u(x)}{dx} = b \cdot cos(bx)\] and
\[\frac {d (v(x))}{dx} = -\frac {d (cos(ax))}{dx} \to\ v(x) = -cos(ax)\]
5. Substitute the terms in \#4 into the Product Rule and integrate both
sides: \[I = \frac{1}{a}\int sin(bx) (\frac {-d(cos(ax))}{dx}){dx}\]

is equivalent to the following in the Product Rule form:

\[\frac {1}{a} \int \frac {d}{dx} (u(x) \cdot v(x)) = \frac {1}{a} \int v(x) \frac {du(x)}{dx} + \frac {1}{a} \int u(x) \frac {dv(x)}{dx}\]

becomes

\[\frac {-1}{a} sin(bx) cos(ax) = \frac {1}{a} \int sin(bx) \left(\frac {-d(cos(ax))}{dx} \right) dx + \frac {1}{a} b \cdot \int -cos(ax) cos(bx){dx}\]
6. Rearrange the terms so that the original integral is isolated:
\[\frac {1}{a} \int sin(bx) \frac {d(-cos(ax))}{dx} = \frac {1}{a} sin(bx) cos(ax) + C_1 - \frac {b}{a} \int cos(bx) cos(ax) dx \]
or
\[I = \frac {1}{a} sin(bx) cos(ax) + C_1 - \frac {b}{a} \int cos(bx) cos(ax) dx \]
7. Repeat the previous steps for the integral on the right-hand side
using the Product Rule again.\\
\[\frac {b}{a} \int cos(bx) cos(ax) dx \] Notice that
\[\frac {d (sin(ax))}{dx} = a \cdot cos(ax)\] so
\[cos(ax) = \frac {1}{a} \frac {d(sin(ax))}{dx}\] and now the integral
becomes
\[\frac {b}{a} \int cos(bx) cos(ax) dx = \frac {b}{a^2}\int  cos(bx) \frac {d(sin(ax))}{dx} dx \]
8. Create the following identities
\[u(x) = cos(bx) \to\ \frac{d u(x)}{dx} = -b \cdot sin(bx)\] and
\[\frac {d (v(x))}{dx} = \frac {d (sin(ax))}{dx} \to\ v(x) = sin(ax)\]
9. Using the Product Rule again
\[\frac {b}{a^2} \int cos(bx) cos(ax) dx \] becomes
\[\frac{b}{a^2} [ cos(bx) sin(ax) + C_1  - b \int sin(ax) sin(bx) dx] \]
Since \$I = \int sin(ax) sin(bx) dx \$ then,
\[\frac{b}{a^2} [ cos(bx) sin(ax) + C_1  - b \int sin(ax) sin(bx) dx] \rightarrow \frac{b}{a^2} [ sin(ax) cos(bx) - b I  + C_2]\]
10. So, from \#6,
\[I = \frac {1}{a} sin(bx) cos(ax) + C_2 - \frac {b}{a} \int cos(bx) cos(ax) dx \]
is now
\[I = \frac {1}{a} \left [\frac {b}{a} \left(cos(ax) sin(bx) + C_2  + b I \right) - sin(bx)cos(ax)\right]\]
\[I = \frac {b}{a^2} \left( sin(ax) cos(bx) + b I \right) - \frac {1}{a}sin(bx)cos(ax) + C_3\]

where \({C_3}\) is the sum of all of the dangling constants.

\begin{enumerate}
\def\labelenumi{\arabic{enumi}.}
\setcounter{enumi}{10}
\tightlist
\item
  Rearrange terms to isolate I
  \[\frac{a^2-b^2}{a^2} I = \frac{b}{a^2} sin(ax) cos(bx) - \frac {1}{a}cos(ax) sin(bx) + C_3\]
  or
  \[I = \frac{a^2}{a^2-b^2} \left( \frac{b}{a^2} sin(ax) cos(bx) - \frac {1}{a}cos(ax) sin(bx) + C_3\right)\]
  and simplifying
  \[I = \frac{a^2}{a^2-b^2} \left( \frac {b \cdot sin(ax) cos(bx) - a \cdot cos(ax) sin(bx)}{a^2} + C_3 \right)\]
\end{enumerate}

\[I = \frac{1}{a^2-b^2} \left(b \cdot sin(ax) cos(bx) - a \cdot cos(ax) sin(bx) + C_3 \right)\]

NOTE that \[{a^2} \neq {b^2}\] 12. Use the following trigonmetric
identities to simplify the right-hand side further
\[cos(a)sin(b) = \frac{1}{2}sin(a+b) - \frac{1}{2}sin(a-b)\]
\[sin(a)cos(b) = \frac{1}{2}sin(a+b) + \frac{1}{2}sin(a-b)\] 13. Rewrite
\(b \cdot sin(ax) cos(bx) - a \cdot cos(ax) sin(bx)\) as follows:
\[\frac{b}{2} \left(sin(a+b)x + sin(a-b)x \right) -\frac{a}{2} \left(sin(a+b)x + sin(a-b)x \right)\]

\begin{enumerate}
\def\labelenumi{\arabic{enumi}.}
\setcounter{enumi}{12}
\item
  Rearrange the terms
  \[\frac{1}{2} \left[(b-a) \cdot sin(a+b)x + (a+b) \cdot sin(a-b)x \right)\]
  and I becomes
  \[I = \frac{1}{a^2 - b^2} \left(\frac{1}{2} \left[(b-a) \cdot sin(a+b)x + (a+b) \cdot sin(a-b)x \right] \right) + {C_4}\]
\item
  Since \({a^2} - {b^2} = (a-b)(a+b)\) and \(b-a = - \left(a-b\right)\)
  \[I = \frac{sin(a-b)x}{2(a-b)} - \frac{sin(a+b)x}{2(a+b)}\]
\end{enumerate}

We can drop the constant \({C_4}\) when the integral is evaluated
between limits.

    \begin{tcolorbox}[breakable, size=fbox, boxrule=1pt, pad at break*=1mm,colback=cellbackground, colframe=cellborder]
\prompt{In}{incolor}{ }{\boxspacing}
\begin{Verbatim}[commandchars=\\\{\}]

\end{Verbatim}
\end{tcolorbox}


    % Add a bibliography block to the postdoc
    
    
    
\end{document}
